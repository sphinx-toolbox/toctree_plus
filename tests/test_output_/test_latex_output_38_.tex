%% Generated by Sphinx.
\def\sphinxdocclass{report}
\documentclass[letterpaper,10pt,english]{sphinxmanual}
\ifdefined\pdfpxdimen
   \let\sphinxpxdimen\pdfpxdimen\else\newdimen\sphinxpxdimen
\fi \sphinxpxdimen=.75bp\relax

\PassOptionsToPackage{warn}{textcomp}
\usepackage[utf8]{inputenc}
\ifdefined\DeclareUnicodeCharacter
% support both utf8 and utf8x syntaxes
  \ifdefined\DeclareUnicodeCharacterAsOptional
    \def\sphinxDUC#1{\DeclareUnicodeCharacter{"#1}}
  \else
    \let\sphinxDUC\DeclareUnicodeCharacter
  \fi
  \sphinxDUC{00A0}{\nobreakspace}
  \sphinxDUC{2500}{\sphinxunichar{2500}}
  \sphinxDUC{2502}{\sphinxunichar{2502}}
  \sphinxDUC{2514}{\sphinxunichar{2514}}
  \sphinxDUC{251C}{\sphinxunichar{251C}}
  \sphinxDUC{2572}{\textbackslash}
\fi
\usepackage{cmap}
\usepackage[T1]{fontenc}
\usepackage{amsmath,amssymb,amstext}
\usepackage{babel}



\usepackage{times}
\expandafter\ifx\csname T@LGR\endcsname\relax
\else
% LGR was declared as font encoding
  \substitutefont{LGR}{\rmdefault}{cmr}
  \substitutefont{LGR}{\sfdefault}{cmss}
  \substitutefont{LGR}{\ttdefault}{cmtt}
\fi
\expandafter\ifx\csname T@X2\endcsname\relax
  \expandafter\ifx\csname T@T2A\endcsname\relax
  \else
  % T2A was declared as font encoding
    \substitutefont{T2A}{\rmdefault}{cmr}
    \substitutefont{T2A}{\sfdefault}{cmss}
    \substitutefont{T2A}{\ttdefault}{cmtt}
  \fi
\else
% X2 was declared as font encoding
  \substitutefont{X2}{\rmdefault}{cmr}
  \substitutefont{X2}{\sfdefault}{cmss}
  \substitutefont{X2}{\ttdefault}{cmtt}
\fi


\usepackage[Bjarne]{fncychap}
\usepackage{sphinx}

\fvset{fontsize=\small}
\usepackage{geometry}


% Include hyperref last.
\usepackage{hyperref}
% Fix anchor placement for figures with captions.
\usepackage{hypcap}% it must be loaded after hyperref.
% Set up styles of URL: it should be placed after hyperref.
\urlstyle{same}

\addto\captionsenglish{\renewcommand{\contentsname}{Contents}}

\usepackage{sphinxmessages}




\title{Python}
\date{Mar 11, 2021}
\release{}
\author{unknown}
\newcommand{\sphinxlogo}{\vbox{}}
\renewcommand{\releasename}{}
\makeindex
\begin{document}

\pagestyle{empty}
\sphinxmaketitle
\pagestyle{plain}
\sphinxtableofcontents
\pagestyle{normal}
\phantomsection\label{\detokenize{index::doc}}



\chapter{\sphinxstyleliteralintitle{\sphinxupquote{csv}} — CSV File Reading and Writing}
\label{\detokenize{csv:module-csv}}\label{\detokenize{csv:csv-csv-file-reading-and-writing}}\label{\detokenize{csv::doc}}\index{module@\spxentry{module}!csv@\spxentry{csv}}\index{csv@\spxentry{csv}!module@\spxentry{module}}
\index{csv@\spxentry{csv}}\index{data@\spxentry{data}!tabular@\spxentry{tabular}}\index{tabular@\spxentry{tabular}!data@\spxentry{data}}\ignorespaces

\bigskip\hrule\bigskip


The so\sphinxhyphen{}called CSV (Comma Separated Values) format is the most common import and
export format for spreadsheets and databases.  CSV format was used for many
years prior to attempts to describe the format in a standardized way in
\index{RFC@\spxentry{RFC}!RFC 4180@\spxentry{RFC 4180}}\sphinxhref{https://tools.ietf.org/html/rfc4180.html}{\sphinxstylestrong{RFC 4180}}.  The lack of a well\sphinxhyphen{}defined standard means that subtle differences
often exist in the data produced and consumed by different applications.  These
differences can make it annoying to process CSV files from multiple sources.
Still, while the delimiters and quoting characters vary, the overall format is
similar enough that it is possible to write a single module which can
efficiently manipulate such data, hiding the details of reading and writing the
data from the programmer.

The {\hyperref[\detokenize{csv:module-csv}]{\sphinxcrossref{\sphinxcode{\sphinxupquote{csv}}}}} module implements classes to read and write tabular data in CSV
format.  It allows programmers to say, “write this data in the format preferred
by Excel,” or “read data from this file which was generated by Excel,” without
knowing the precise details of the CSV format used by Excel.  Programmers can
also describe the CSV formats understood by other applications or define their
own special\sphinxhyphen{}purpose CSV formats.

The {\hyperref[\detokenize{csv:module-csv}]{\sphinxcrossref{\sphinxcode{\sphinxupquote{csv}}}}} module’s {\hyperref[\detokenize{csv:csv.reader}]{\sphinxcrossref{\sphinxcode{\sphinxupquote{reader}}}}} and {\hyperref[\detokenize{csv:csv.writer}]{\sphinxcrossref{\sphinxcode{\sphinxupquote{writer}}}}} objects read and
write sequences.  Programmers can also read and write data in dictionary form
using the {\hyperref[\detokenize{csv:csv.DictReader}]{\sphinxcrossref{\sphinxcode{\sphinxupquote{DictReader}}}}} and {\hyperref[\detokenize{csv:csv.DictWriter}]{\sphinxcrossref{\sphinxcode{\sphinxupquote{DictWriter}}}}} classes.


\section{Module Contents}
\label{\detokenize{csv:module-contents}}\label{\detokenize{csv:csv-contents}}
The {\hyperref[\detokenize{csv:module-csv}]{\sphinxcrossref{\sphinxcode{\sphinxupquote{csv}}}}} module defines the following functions:

\index{universal newlines@\spxentry{universal newlines}!csv.reader function@\spxentry{csv.reader function}}\ignorespaces \index{reader() (in module csv)@\spxentry{reader()}\spxextra{in module csv}}

\vspace{5px}

\begin{fulllineitems}
\phantomsection\stepcounter{subsection}
\addcontentsline{toc}{subsection}{\protect\numberline{\thesubsection}{reader}}
\phantomsection\label{\detokenize{csv:csv.reader}}\pysiglinewithargsret{\sphinxcode{\sphinxupquote{csv.}}\sphinxbfcode{\sphinxupquote{reader}}}{\emph{\DUrole{n}{csvfile}}, \emph{\DUrole{n}{dialect}\DUrole{o}{=}\DUrole{default_value}{\textquotesingle{}excel\textquotesingle{}}}, \emph{\DUrole{o}{**}\DUrole{n}{fmtparams}}}{}
Return a reader object which will iterate over lines in the given \sphinxstyleemphasis{csvfile}.
\sphinxstyleemphasis{csvfile} can be any object which supports the \DUrole{xref,std,std-term}{iterator} protocol and returns a
string each time its \sphinxcode{\sphinxupquote{\_\_next\_\_()}} method is called — \DUrole{xref,std,std-term}{file objects} and list objects are both suitable.   If \sphinxstyleemphasis{csvfile} is a file object,
it should be opened with \sphinxcode{\sphinxupquote{newline=\textquotesingle{}\textquotesingle{}}}. %
\begin{footnote}[1]\sphinxAtStartFootnote
If \sphinxcode{\sphinxupquote{newline=\textquotesingle{}\textquotesingle{}}} is not specified, newlines embedded inside quoted fields
will not be interpreted correctly, and on platforms that use \sphinxcode{\sphinxupquote{\textbackslash{}r\textbackslash{}n}} linendings
on write an extra \sphinxcode{\sphinxupquote{\textbackslash{}r}} will be added.  It should always be safe to specify
\sphinxcode{\sphinxupquote{newline=\textquotesingle{}\textquotesingle{}}}, since the csv module does its own
(\DUrole{xref,std,std-term}{universal}) newline handling.
%
\end{footnote}  An optional
\sphinxstyleemphasis{dialect} parameter can be given which is used to define a set of parameters
specific to a particular CSV dialect.  It may be an instance of a subclass of
the {\hyperref[\detokenize{csv:csv.Dialect}]{\sphinxcrossref{\sphinxcode{\sphinxupquote{Dialect}}}}} class or one of the strings returned by the
{\hyperref[\detokenize{csv:csv.list_dialects}]{\sphinxcrossref{\sphinxcode{\sphinxupquote{list\_dialects()}}}}} function.  The other optional \sphinxstyleemphasis{fmtparams} keyword arguments
can be given to override individual formatting parameters in the current
dialect.  For full details about the dialect and formatting parameters, see
section {\hyperref[\detokenize{csv:csv-fmt-params}]{\sphinxcrossref{\DUrole{std,std-ref}{Dialects and Formatting Parameters}}}}.

Each row read from the csv file is returned as a list of strings.  No
automatic data type conversion is performed unless the \sphinxcode{\sphinxupquote{QUOTE\_NONNUMERIC}} format
option is specified (in which case unquoted fields are transformed into floats).

A short usage example:

\begin{sphinxVerbatim}[commandchars=\\\{\}]
\PYG{g+gp}{\PYGZgt{}\PYGZgt{}\PYGZgt{} }\PYG{k+kn}{import} \PYG{n+nn}{csv}
\PYG{g+gp}{\PYGZgt{}\PYGZgt{}\PYGZgt{} }\PYG{k}{with} \PYG{n+nb}{open}\PYG{p}{(}\PYG{l+s+s1}{\PYGZsq{}}\PYG{l+s+s1}{eggs.csv}\PYG{l+s+s1}{\PYGZsq{}}\PYG{p}{,} \PYG{n}{newline}\PYG{o}{=}\PYG{l+s+s1}{\PYGZsq{}}\PYG{l+s+s1}{\PYGZsq{}}\PYG{p}{)} \PYG{k}{as} \PYG{n}{csvfile}\PYG{p}{:}
\PYG{g+gp}{... }    \PYG{n}{spamreader} \PYG{o}{=} \PYG{n}{csv}\PYG{o}{.}\PYG{n}{reader}\PYG{p}{(}\PYG{n}{csvfile}\PYG{p}{,} \PYG{n}{delimiter}\PYG{o}{=}\PYG{l+s+s1}{\PYGZsq{}}\PYG{l+s+s1}{ }\PYG{l+s+s1}{\PYGZsq{}}\PYG{p}{,} \PYG{n}{quotechar}\PYG{o}{=}\PYG{l+s+s1}{\PYGZsq{}}\PYG{l+s+s1}{|}\PYG{l+s+s1}{\PYGZsq{}}\PYG{p}{)}
\PYG{g+gp}{... }    \PYG{k}{for} \PYG{n}{row} \PYG{o+ow}{in} \PYG{n}{spamreader}\PYG{p}{:}
\PYG{g+gp}{... }        \PYG{n+nb}{print}\PYG{p}{(}\PYG{l+s+s1}{\PYGZsq{}}\PYG{l+s+s1}{, }\PYG{l+s+s1}{\PYGZsq{}}\PYG{o}{.}\PYG{n}{join}\PYG{p}{(}\PYG{n}{row}\PYG{p}{)}\PYG{p}{)}
\PYG{g+go}{Spam, Spam, Spam, Spam, Spam, Baked Beans}
\PYG{g+go}{Spam, Lovely Spam, Wonderful Spam}
\end{sphinxVerbatim}

\end{fulllineitems}

\index{writer() (in module csv)@\spxentry{writer()}\spxextra{in module csv}}

\vspace{5px}

\begin{fulllineitems}
\phantomsection\stepcounter{subsection}
\addcontentsline{toc}{subsection}{\protect\numberline{\thesubsection}{writer}}
\phantomsection\label{\detokenize{csv:csv.writer}}\pysiglinewithargsret{\sphinxcode{\sphinxupquote{csv.}}\sphinxbfcode{\sphinxupquote{writer}}}{\emph{\DUrole{n}{csvfile}}, \emph{\DUrole{n}{dialect}\DUrole{o}{=}\DUrole{default_value}{\textquotesingle{}excel\textquotesingle{}}}, \emph{\DUrole{o}{**}\DUrole{n}{fmtparams}}}{}
Return a writer object responsible for converting the user’s data into delimited
strings on the given file\sphinxhyphen{}like object.  \sphinxstyleemphasis{csvfile} can be any object with a
\sphinxcode{\sphinxupquote{write()}} method.  If \sphinxstyleemphasis{csvfile} is a file object, it should be opened with
\sphinxcode{\sphinxupquote{newline=\textquotesingle{}\textquotesingle{}}} \sphinxfootnotemark[1].  An optional \sphinxstyleemphasis{dialect}
parameter can be given which is used to define a set of parameters specific to a
particular CSV dialect.  It may be an instance of a subclass of the
{\hyperref[\detokenize{csv:csv.Dialect}]{\sphinxcrossref{\sphinxcode{\sphinxupquote{Dialect}}}}} class or one of the strings returned by the
{\hyperref[\detokenize{csv:csv.list_dialects}]{\sphinxcrossref{\sphinxcode{\sphinxupquote{list\_dialects()}}}}} function.  The other optional \sphinxstyleemphasis{fmtparams} keyword arguments
can be given to override individual formatting parameters in the current
dialect.  For full details about the dialect and formatting parameters, see
section {\hyperref[\detokenize{csv:csv-fmt-params}]{\sphinxcrossref{\DUrole{std,std-ref}{Dialects and Formatting Parameters}}}}. To make it
as easy as possible to interface with modules which implement the DB API, the
value \sphinxcode{\sphinxupquote{None}} is written as the empty string.  While this isn’t a
reversible transformation, it makes it easier to dump SQL NULL data values to
CSV files without preprocessing the data returned from a \sphinxcode{\sphinxupquote{cursor.fetch*}} call.
All other non\sphinxhyphen{}string data are stringified with \sphinxcode{\sphinxupquote{str()}} before being written.

A short usage example:

\begin{sphinxVerbatim}[commandchars=\\\{\}]
\PYG{k+kn}{import} \PYG{n+nn}{csv}
\PYG{k}{with} \PYG{n+nb}{open}\PYG{p}{(}\PYG{l+s+s1}{\PYGZsq{}}\PYG{l+s+s1}{eggs.csv}\PYG{l+s+s1}{\PYGZsq{}}\PYG{p}{,} \PYG{l+s+s1}{\PYGZsq{}}\PYG{l+s+s1}{w}\PYG{l+s+s1}{\PYGZsq{}}\PYG{p}{,} \PYG{n}{newline}\PYG{o}{=}\PYG{l+s+s1}{\PYGZsq{}}\PYG{l+s+s1}{\PYGZsq{}}\PYG{p}{)} \PYG{k}{as} \PYG{n}{csvfile}\PYG{p}{:}
    \PYG{n}{spamwriter} \PYG{o}{=} \PYG{n}{csv}\PYG{o}{.}\PYG{n}{writer}\PYG{p}{(}\PYG{n}{csvfile}\PYG{p}{,} \PYG{n}{delimiter}\PYG{o}{=}\PYG{l+s+s1}{\PYGZsq{}}\PYG{l+s+s1}{ }\PYG{l+s+s1}{\PYGZsq{}}\PYG{p}{,}
                            \PYG{n}{quotechar}\PYG{o}{=}\PYG{l+s+s1}{\PYGZsq{}}\PYG{l+s+s1}{|}\PYG{l+s+s1}{\PYGZsq{}}\PYG{p}{,} \PYG{n}{quoting}\PYG{o}{=}\PYG{n}{csv}\PYG{o}{.}\PYG{n}{QUOTE\PYGZus{}MINIMAL}\PYG{p}{)}
    \PYG{n}{spamwriter}\PYG{o}{.}\PYG{n}{writerow}\PYG{p}{(}\PYG{p}{[}\PYG{l+s+s1}{\PYGZsq{}}\PYG{l+s+s1}{Spam}\PYG{l+s+s1}{\PYGZsq{}}\PYG{p}{]} \PYG{o}{*} \PYG{l+m+mi}{5} \PYG{o}{+} \PYG{p}{[}\PYG{l+s+s1}{\PYGZsq{}}\PYG{l+s+s1}{Baked Beans}\PYG{l+s+s1}{\PYGZsq{}}\PYG{p}{]}\PYG{p}{)}
    \PYG{n}{spamwriter}\PYG{o}{.}\PYG{n}{writerow}\PYG{p}{(}\PYG{p}{[}\PYG{l+s+s1}{\PYGZsq{}}\PYG{l+s+s1}{Spam}\PYG{l+s+s1}{\PYGZsq{}}\PYG{p}{,} \PYG{l+s+s1}{\PYGZsq{}}\PYG{l+s+s1}{Lovely Spam}\PYG{l+s+s1}{\PYGZsq{}}\PYG{p}{,} \PYG{l+s+s1}{\PYGZsq{}}\PYG{l+s+s1}{Wonderful Spam}\PYG{l+s+s1}{\PYGZsq{}}\PYG{p}{]}\PYG{p}{)}
\end{sphinxVerbatim}

\end{fulllineitems}

\index{register\_dialect() (in module csv)@\spxentry{register\_dialect()}\spxextra{in module csv}}

\vspace{5px}

\begin{fulllineitems}
\phantomsection\stepcounter{subsection}
\addcontentsline{toc}{subsection}{\protect\numberline{\thesubsection}{register\_dialect}}
\phantomsection\label{\detokenize{csv:csv.register_dialect}}\pysiglinewithargsret{\sphinxcode{\sphinxupquote{csv.}}\sphinxbfcode{\sphinxupquote{register\_dialect}}}{\emph{name}\sphinxoptional{, \emph{dialect}\sphinxoptional{, \emph{**fmtparams}}}}{}
Associate \sphinxstyleemphasis{dialect} with \sphinxstyleemphasis{name}.  \sphinxstyleemphasis{name} must be a string. The
dialect can be specified either by passing a sub\sphinxhyphen{}class of {\hyperref[\detokenize{csv:csv.Dialect}]{\sphinxcrossref{\sphinxcode{\sphinxupquote{Dialect}}}}}, or
by \sphinxstyleemphasis{fmtparams} keyword arguments, or both, with keyword arguments overriding
parameters of the dialect. For full details about the dialect and formatting
parameters, see section {\hyperref[\detokenize{csv:csv-fmt-params}]{\sphinxcrossref{\DUrole{std,std-ref}{Dialects and Formatting Parameters}}}}.

\end{fulllineitems}

\index{unregister\_dialect() (in module csv)@\spxentry{unregister\_dialect()}\spxextra{in module csv}}

\vspace{5px}

\begin{fulllineitems}
\phantomsection\stepcounter{subsection}
\addcontentsline{toc}{subsection}{\protect\numberline{\thesubsection}{unregister\_dialect}}
\phantomsection\label{\detokenize{csv:csv.unregister_dialect}}\pysiglinewithargsret{\sphinxcode{\sphinxupquote{csv.}}\sphinxbfcode{\sphinxupquote{unregister\_dialect}}}{\emph{\DUrole{n}{name}}}{}
Delete the dialect associated with \sphinxstyleemphasis{name} from the dialect registry.  An
{\hyperref[\detokenize{csv:csv.Error}]{\sphinxcrossref{\sphinxcode{\sphinxupquote{Error}}}}} is raised if \sphinxstyleemphasis{name} is not a registered dialect name.

\end{fulllineitems}

\index{get\_dialect() (in module csv)@\spxentry{get\_dialect()}\spxextra{in module csv}}

\vspace{5px}

\begin{fulllineitems}
\phantomsection\stepcounter{subsection}
\addcontentsline{toc}{subsection}{\protect\numberline{\thesubsection}{get\_dialect}}
\phantomsection\label{\detokenize{csv:csv.get_dialect}}\pysiglinewithargsret{\sphinxcode{\sphinxupquote{csv.}}\sphinxbfcode{\sphinxupquote{get\_dialect}}}{\emph{\DUrole{n}{name}}}{}
Return the dialect associated with \sphinxstyleemphasis{name}.  An {\hyperref[\detokenize{csv:csv.Error}]{\sphinxcrossref{\sphinxcode{\sphinxupquote{Error}}}}} is raised if
\sphinxstyleemphasis{name} is not a registered dialect name.  This function returns an immutable
{\hyperref[\detokenize{csv:csv.Dialect}]{\sphinxcrossref{\sphinxcode{\sphinxupquote{Dialect}}}}}.

\end{fulllineitems}

\index{list\_dialects() (in module csv)@\spxentry{list\_dialects()}\spxextra{in module csv}}

\vspace{5px}

\begin{fulllineitems}
\phantomsection\stepcounter{subsection}
\addcontentsline{toc}{subsection}{\protect\numberline{\thesubsection}{list\_dialects}}
\phantomsection\label{\detokenize{csv:csv.list_dialects}}\pysiglinewithargsret{\sphinxcode{\sphinxupquote{csv.}}\sphinxbfcode{\sphinxupquote{list\_dialects}}}{}{}
Return the names of all registered dialects.

\end{fulllineitems}

\index{field\_size\_limit() (in module csv)@\spxentry{field\_size\_limit()}\spxextra{in module csv}}

\vspace{5px}

\begin{fulllineitems}
\phantomsection\stepcounter{subsection}
\addcontentsline{toc}{subsection}{\protect\numberline{\thesubsection}{field\_size\_limit}}
\phantomsection\label{\detokenize{csv:csv.field_size_limit}}\pysiglinewithargsret{\sphinxcode{\sphinxupquote{csv.}}\sphinxbfcode{\sphinxupquote{field\_size\_limit}}}{\sphinxoptional{\emph{new\_limit}}}{}
Returns the current maximum field size allowed by the parser. If \sphinxstyleemphasis{new\_limit} is
given, this becomes the new limit.

\end{fulllineitems}


The {\hyperref[\detokenize{csv:module-csv}]{\sphinxcrossref{\sphinxcode{\sphinxupquote{csv}}}}} module defines the following classes:
\index{DictReader (class in csv)@\spxentry{DictReader}\spxextra{class in csv}}

\vspace{5px}

\begin{fulllineitems}
\phantomsection\stepcounter{subsection}
\addcontentsline{toc}{subsection}{\protect\numberline{\thesubsection}{DictReader}}
\phantomsection\label{\detokenize{csv:csv.DictReader}}\pysiglinewithargsret{\sphinxbfcode{\sphinxupquote{class }}\sphinxcode{\sphinxupquote{csv.}}\sphinxbfcode{\sphinxupquote{DictReader}}}{\emph{\DUrole{n}{f}}, \emph{\DUrole{n}{fieldnames}\DUrole{o}{=}\DUrole{default_value}{None}}, \emph{\DUrole{n}{restkey}\DUrole{o}{=}\DUrole{default_value}{None}}, \emph{\DUrole{n}{restval}\DUrole{o}{=}\DUrole{default_value}{None}}, \emph{\DUrole{n}{dialect}\DUrole{o}{=}\DUrole{default_value}{\textquotesingle{}excel\textquotesingle{}}}, \emph{\DUrole{o}{*}\DUrole{n}{args}}, \emph{\DUrole{o}{**}\DUrole{n}{kwds}}}{}
Create an object that operates like a regular reader but maps the
information in each row to a \sphinxcode{\sphinxupquote{dict}} whose keys are given by the
optional \sphinxstyleemphasis{fieldnames} parameter.

The \sphinxstyleemphasis{fieldnames} parameter is a \DUrole{xref,std,std-term}{sequence}.  If \sphinxstyleemphasis{fieldnames} is
omitted, the values in the first row of file \sphinxstyleemphasis{f} will be used as the
fieldnames.  Regardless of how the fieldnames are determined, the
dictionary preserves their original ordering.

If a row has more fields than fieldnames, the remaining data is put in a
list and stored with the fieldname specified by \sphinxstyleemphasis{restkey} (which defaults
to \sphinxcode{\sphinxupquote{None}}).  If a non\sphinxhyphen{}blank row has fewer fields than fieldnames, the
missing values are filled\sphinxhyphen{}in with the value of \sphinxstyleemphasis{restval} (which defaults
to \sphinxcode{\sphinxupquote{None}}).

All other optional or keyword arguments are passed to the underlying
{\hyperref[\detokenize{csv:csv.reader}]{\sphinxcrossref{\sphinxcode{\sphinxupquote{reader}}}}} instance.

\DUrole{versionmodified,changed}{Changed in version 3.6: }Returned rows are now of type \sphinxcode{\sphinxupquote{OrderedDict}}.

\DUrole{versionmodified,changed}{Changed in version 3.8: }Returned rows are now of type \sphinxcode{\sphinxupquote{dict}}.

A short usage example:

\begin{sphinxVerbatim}[commandchars=\\\{\}]
\PYG{g+gp}{\PYGZgt{}\PYGZgt{}\PYGZgt{} }\PYG{k+kn}{import} \PYG{n+nn}{csv}
\PYG{g+gp}{\PYGZgt{}\PYGZgt{}\PYGZgt{} }\PYG{k}{with} \PYG{n+nb}{open}\PYG{p}{(}\PYG{l+s+s1}{\PYGZsq{}}\PYG{l+s+s1}{names.csv}\PYG{l+s+s1}{\PYGZsq{}}\PYG{p}{,} \PYG{n}{newline}\PYG{o}{=}\PYG{l+s+s1}{\PYGZsq{}}\PYG{l+s+s1}{\PYGZsq{}}\PYG{p}{)} \PYG{k}{as} \PYG{n}{csvfile}\PYG{p}{:}
\PYG{g+gp}{... }    \PYG{n}{reader} \PYG{o}{=} \PYG{n}{csv}\PYG{o}{.}\PYG{n}{DictReader}\PYG{p}{(}\PYG{n}{csvfile}\PYG{p}{)}
\PYG{g+gp}{... }    \PYG{k}{for} \PYG{n}{row} \PYG{o+ow}{in} \PYG{n}{reader}\PYG{p}{:}
\PYG{g+gp}{... }        \PYG{n+nb}{print}\PYG{p}{(}\PYG{n}{row}\PYG{p}{[}\PYG{l+s+s1}{\PYGZsq{}}\PYG{l+s+s1}{first\PYGZus{}name}\PYG{l+s+s1}{\PYGZsq{}}\PYG{p}{]}\PYG{p}{,} \PYG{n}{row}\PYG{p}{[}\PYG{l+s+s1}{\PYGZsq{}}\PYG{l+s+s1}{last\PYGZus{}name}\PYG{l+s+s1}{\PYGZsq{}}\PYG{p}{]}\PYG{p}{)}
\PYG{g+gp}{...}
\PYG{g+go}{Eric Idle}
\PYG{g+go}{John Cleese}

\PYG{g+gp}{\PYGZgt{}\PYGZgt{}\PYGZgt{} }\PYG{n+nb}{print}\PYG{p}{(}\PYG{n}{row}\PYG{p}{)}
\PYG{g+go}{\PYGZob{}\PYGZsq{}first\PYGZus{}name\PYGZsq{}: \PYGZsq{}John\PYGZsq{}, \PYGZsq{}last\PYGZus{}name\PYGZsq{}: \PYGZsq{}Cleese\PYGZsq{}\PYGZcb{}}
\end{sphinxVerbatim}

\end{fulllineitems}

\index{DictWriter (class in csv)@\spxentry{DictWriter}\spxextra{class in csv}}

\vspace{5px}

\begin{fulllineitems}
\phantomsection\stepcounter{subsection}
\addcontentsline{toc}{subsection}{\protect\numberline{\thesubsection}{DictWriter}}
\phantomsection\label{\detokenize{csv:csv.DictWriter}}\pysiglinewithargsret{\sphinxbfcode{\sphinxupquote{class }}\sphinxcode{\sphinxupquote{csv.}}\sphinxbfcode{\sphinxupquote{DictWriter}}}{\emph{\DUrole{n}{f}}, \emph{\DUrole{n}{fieldnames}}, \emph{\DUrole{n}{restval}\DUrole{o}{=}\DUrole{default_value}{\textquotesingle{}\textquotesingle{}}}, \emph{\DUrole{n}{extrasaction}\DUrole{o}{=}\DUrole{default_value}{\textquotesingle{}raise\textquotesingle{}}}, \emph{\DUrole{n}{dialect}\DUrole{o}{=}\DUrole{default_value}{\textquotesingle{}excel\textquotesingle{}}}, \emph{\DUrole{o}{*}\DUrole{n}{args}}, \emph{\DUrole{o}{**}\DUrole{n}{kwds}}}{}
Create an object which operates like a regular writer but maps dictionaries
onto output rows.  The \sphinxstyleemphasis{fieldnames} parameter is a \sphinxcode{\sphinxupquote{sequence}} of keys that identify the order in which values in the
dictionary passed to the \sphinxcode{\sphinxupquote{writerow()}} method are written to file
\sphinxstyleemphasis{f}.  The optional \sphinxstyleemphasis{restval} parameter specifies the value to be
written if the dictionary is missing a key in \sphinxstyleemphasis{fieldnames}.  If the
dictionary passed to the \sphinxcode{\sphinxupquote{writerow()}} method contains a key not found in
\sphinxstyleemphasis{fieldnames}, the optional \sphinxstyleemphasis{extrasaction} parameter indicates what action to
take.
If it is set to \sphinxcode{\sphinxupquote{\textquotesingle{}raise\textquotesingle{}}}, the default value, a \sphinxcode{\sphinxupquote{ValueError}}
is raised.
If it is set to \sphinxcode{\sphinxupquote{\textquotesingle{}ignore\textquotesingle{}}}, extra values in the dictionary are ignored.
Any other optional or keyword arguments are passed to the underlying
{\hyperref[\detokenize{csv:csv.writer}]{\sphinxcrossref{\sphinxcode{\sphinxupquote{writer}}}}} instance.

Note that unlike the {\hyperref[\detokenize{csv:csv.DictReader}]{\sphinxcrossref{\sphinxcode{\sphinxupquote{DictReader}}}}} class, the \sphinxstyleemphasis{fieldnames} parameter
of the {\hyperref[\detokenize{csv:csv.DictWriter}]{\sphinxcrossref{\sphinxcode{\sphinxupquote{DictWriter}}}}} class is not optional.

A short usage example:

\begin{sphinxVerbatim}[commandchars=\\\{\}]
\PYG{k+kn}{import} \PYG{n+nn}{csv}

\PYG{k}{with} \PYG{n+nb}{open}\PYG{p}{(}\PYG{l+s+s1}{\PYGZsq{}}\PYG{l+s+s1}{names.csv}\PYG{l+s+s1}{\PYGZsq{}}\PYG{p}{,} \PYG{l+s+s1}{\PYGZsq{}}\PYG{l+s+s1}{w}\PYG{l+s+s1}{\PYGZsq{}}\PYG{p}{,} \PYG{n}{newline}\PYG{o}{=}\PYG{l+s+s1}{\PYGZsq{}}\PYG{l+s+s1}{\PYGZsq{}}\PYG{p}{)} \PYG{k}{as} \PYG{n}{csvfile}\PYG{p}{:}
    \PYG{n}{fieldnames} \PYG{o}{=} \PYG{p}{[}\PYG{l+s+s1}{\PYGZsq{}}\PYG{l+s+s1}{first\PYGZus{}name}\PYG{l+s+s1}{\PYGZsq{}}\PYG{p}{,} \PYG{l+s+s1}{\PYGZsq{}}\PYG{l+s+s1}{last\PYGZus{}name}\PYG{l+s+s1}{\PYGZsq{}}\PYG{p}{]}
    \PYG{n}{writer} \PYG{o}{=} \PYG{n}{csv}\PYG{o}{.}\PYG{n}{DictWriter}\PYG{p}{(}\PYG{n}{csvfile}\PYG{p}{,} \PYG{n}{fieldnames}\PYG{o}{=}\PYG{n}{fieldnames}\PYG{p}{)}

    \PYG{n}{writer}\PYG{o}{.}\PYG{n}{writeheader}\PYG{p}{(}\PYG{p}{)}
    \PYG{n}{writer}\PYG{o}{.}\PYG{n}{writerow}\PYG{p}{(}\PYG{p}{\PYGZob{}}\PYG{l+s+s1}{\PYGZsq{}}\PYG{l+s+s1}{first\PYGZus{}name}\PYG{l+s+s1}{\PYGZsq{}}\PYG{p}{:} \PYG{l+s+s1}{\PYGZsq{}}\PYG{l+s+s1}{Baked}\PYG{l+s+s1}{\PYGZsq{}}\PYG{p}{,} \PYG{l+s+s1}{\PYGZsq{}}\PYG{l+s+s1}{last\PYGZus{}name}\PYG{l+s+s1}{\PYGZsq{}}\PYG{p}{:} \PYG{l+s+s1}{\PYGZsq{}}\PYG{l+s+s1}{Beans}\PYG{l+s+s1}{\PYGZsq{}}\PYG{p}{\PYGZcb{}}\PYG{p}{)}
    \PYG{n}{writer}\PYG{o}{.}\PYG{n}{writerow}\PYG{p}{(}\PYG{p}{\PYGZob{}}\PYG{l+s+s1}{\PYGZsq{}}\PYG{l+s+s1}{first\PYGZus{}name}\PYG{l+s+s1}{\PYGZsq{}}\PYG{p}{:} \PYG{l+s+s1}{\PYGZsq{}}\PYG{l+s+s1}{Lovely}\PYG{l+s+s1}{\PYGZsq{}}\PYG{p}{,} \PYG{l+s+s1}{\PYGZsq{}}\PYG{l+s+s1}{last\PYGZus{}name}\PYG{l+s+s1}{\PYGZsq{}}\PYG{p}{:} \PYG{l+s+s1}{\PYGZsq{}}\PYG{l+s+s1}{Spam}\PYG{l+s+s1}{\PYGZsq{}}\PYG{p}{\PYGZcb{}}\PYG{p}{)}
    \PYG{n}{writer}\PYG{o}{.}\PYG{n}{writerow}\PYG{p}{(}\PYG{p}{\PYGZob{}}\PYG{l+s+s1}{\PYGZsq{}}\PYG{l+s+s1}{first\PYGZus{}name}\PYG{l+s+s1}{\PYGZsq{}}\PYG{p}{:} \PYG{l+s+s1}{\PYGZsq{}}\PYG{l+s+s1}{Wonderful}\PYG{l+s+s1}{\PYGZsq{}}\PYG{p}{,} \PYG{l+s+s1}{\PYGZsq{}}\PYG{l+s+s1}{last\PYGZus{}name}\PYG{l+s+s1}{\PYGZsq{}}\PYG{p}{:} \PYG{l+s+s1}{\PYGZsq{}}\PYG{l+s+s1}{Spam}\PYG{l+s+s1}{\PYGZsq{}}\PYG{p}{\PYGZcb{}}\PYG{p}{)}
\end{sphinxVerbatim}

\end{fulllineitems}

\index{Dialect (class in csv)@\spxentry{Dialect}\spxextra{class in csv}}

\vspace{5px}

\begin{fulllineitems}
\phantomsection\stepcounter{subsection}
\addcontentsline{toc}{subsection}{\protect\numberline{\thesubsection}{Dialect}}
\phantomsection\label{\detokenize{csv:csv.Dialect}}\pysigline{\sphinxbfcode{\sphinxupquote{class }}\sphinxcode{\sphinxupquote{csv.}}\sphinxbfcode{\sphinxupquote{Dialect}}}
The {\hyperref[\detokenize{csv:csv.Dialect}]{\sphinxcrossref{\sphinxcode{\sphinxupquote{Dialect}}}}} class is a container class relied on primarily for its
attributes, which are used to define the parameters for a specific
{\hyperref[\detokenize{csv:csv.reader}]{\sphinxcrossref{\sphinxcode{\sphinxupquote{reader}}}}} or {\hyperref[\detokenize{csv:csv.writer}]{\sphinxcrossref{\sphinxcode{\sphinxupquote{writer}}}}} instance.

\end{fulllineitems}

\index{excel (class in csv)@\spxentry{excel}\spxextra{class in csv}}

\vspace{5px}

\begin{fulllineitems}
\phantomsection\stepcounter{subsection}
\addcontentsline{toc}{subsection}{\protect\numberline{\thesubsection}{excel}}
\phantomsection\label{\detokenize{csv:csv.excel}}\pysigline{\sphinxbfcode{\sphinxupquote{class }}\sphinxcode{\sphinxupquote{csv.}}\sphinxbfcode{\sphinxupquote{excel}}}
The {\hyperref[\detokenize{csv:csv.excel}]{\sphinxcrossref{\sphinxcode{\sphinxupquote{excel}}}}} class defines the usual properties of an Excel\sphinxhyphen{}generated CSV
file.  It is registered with the dialect name \sphinxcode{\sphinxupquote{\textquotesingle{}excel\textquotesingle{}}}.

\end{fulllineitems}

\index{excel\_tab (class in csv)@\spxentry{excel\_tab}\spxextra{class in csv}}

\vspace{5px}

\begin{fulllineitems}
\phantomsection\stepcounter{subsection}
\addcontentsline{toc}{subsection}{\protect\numberline{\thesubsection}{excel\_tab}}
\phantomsection\label{\detokenize{csv:csv.excel_tab}}\pysigline{\sphinxbfcode{\sphinxupquote{class }}\sphinxcode{\sphinxupquote{csv.}}\sphinxbfcode{\sphinxupquote{excel\_tab}}}
The {\hyperref[\detokenize{csv:csv.excel_tab}]{\sphinxcrossref{\sphinxcode{\sphinxupquote{excel\_tab}}}}} class defines the usual properties of an Excel\sphinxhyphen{}generated
TAB\sphinxhyphen{}delimited file.  It is registered with the dialect name \sphinxcode{\sphinxupquote{\textquotesingle{}excel\sphinxhyphen{}tab\textquotesingle{}}}.

\end{fulllineitems}

\index{unix\_dialect (class in csv)@\spxentry{unix\_dialect}\spxextra{class in csv}}

\vspace{5px}

\begin{fulllineitems}
\phantomsection\stepcounter{subsection}
\addcontentsline{toc}{subsection}{\protect\numberline{\thesubsection}{unix\_dialect}}
\phantomsection\label{\detokenize{csv:csv.unix_dialect}}\pysigline{\sphinxbfcode{\sphinxupquote{class }}\sphinxcode{\sphinxupquote{csv.}}\sphinxbfcode{\sphinxupquote{unix\_dialect}}}
The {\hyperref[\detokenize{csv:csv.unix_dialect}]{\sphinxcrossref{\sphinxcode{\sphinxupquote{unix\_dialect}}}}} class defines the usual properties of a CSV file
generated on UNIX systems, i.e. using \sphinxcode{\sphinxupquote{\textquotesingle{}\textbackslash{}n\textquotesingle{}}} as line terminator and quoting
all fields.  It is registered with the dialect name \sphinxcode{\sphinxupquote{\textquotesingle{}unix\textquotesingle{}}}.

\DUrole{versionmodified,added}{New in version 3.2.}

\end{fulllineitems}

\index{Sniffer (class in csv)@\spxentry{Sniffer}\spxextra{class in csv}}

\vspace{5px}

\begin{fulllineitems}
\phantomsection\stepcounter{subsection}
\addcontentsline{toc}{subsection}{\protect\numberline{\thesubsection}{Sniffer}}
\phantomsection\label{\detokenize{csv:csv.Sniffer}}\pysigline{\sphinxbfcode{\sphinxupquote{class }}\sphinxcode{\sphinxupquote{csv.}}\sphinxbfcode{\sphinxupquote{Sniffer}}}
The {\hyperref[\detokenize{csv:csv.Sniffer}]{\sphinxcrossref{\sphinxcode{\sphinxupquote{Sniffer}}}}} class is used to deduce the format of a CSV file.

The {\hyperref[\detokenize{csv:csv.Sniffer}]{\sphinxcrossref{\sphinxcode{\sphinxupquote{Sniffer}}}}} class provides two methods:
\index{sniff() (csv.Sniffer method)@\spxentry{sniff()}\spxextra{csv.Sniffer method}}

\vspace{5px}

\begin{fulllineitems}
\phantomsection\stepcounter{subsubsection}
\addcontentsline{toc}{subsubsection}{\protect\numberline{\thesubsubsection}{sniff}}
\phantomsection\label{\detokenize{csv:csv.Sniffer.sniff}}\pysiglinewithargsret{\sphinxbfcode{\sphinxupquote{sniff}}}{\emph{\DUrole{n}{sample}}, \emph{\DUrole{n}{delimiters}\DUrole{o}{=}\DUrole{default_value}{None}}}{}
Analyze the given \sphinxstyleemphasis{sample} and return a {\hyperref[\detokenize{csv:csv.Dialect}]{\sphinxcrossref{\sphinxcode{\sphinxupquote{Dialect}}}}} subclass
reflecting the parameters found.  If the optional \sphinxstyleemphasis{delimiters} parameter
is given, it is interpreted as a string containing possible valid
delimiter characters.

\end{fulllineitems}

\index{has\_header() (csv.Sniffer method)@\spxentry{has\_header()}\spxextra{csv.Sniffer method}}

\vspace{5px}

\begin{fulllineitems}
\phantomsection\stepcounter{subsubsection}
\addcontentsline{toc}{subsubsection}{\protect\numberline{\thesubsubsection}{has\_header}}
\phantomsection\label{\detokenize{csv:csv.Sniffer.has_header}}\pysiglinewithargsret{\sphinxbfcode{\sphinxupquote{has\_header}}}{\emph{\DUrole{n}{sample}}}{}
Analyze the sample text (presumed to be in CSV format) and return
\sphinxcode{\sphinxupquote{True}} if the first row appears to be a series of column headers.

\end{fulllineitems}


\end{fulllineitems}


An example for {\hyperref[\detokenize{csv:csv.Sniffer}]{\sphinxcrossref{\sphinxcode{\sphinxupquote{Sniffer}}}}} use:

\begin{sphinxVerbatim}[commandchars=\\\{\}]
\PYG{k}{with} \PYG{n+nb}{open}\PYG{p}{(}\PYG{l+s+s1}{\PYGZsq{}}\PYG{l+s+s1}{example.csv}\PYG{l+s+s1}{\PYGZsq{}}\PYG{p}{,} \PYG{n}{newline}\PYG{o}{=}\PYG{l+s+s1}{\PYGZsq{}}\PYG{l+s+s1}{\PYGZsq{}}\PYG{p}{)} \PYG{k}{as} \PYG{n}{csvfile}\PYG{p}{:}
    \PYG{n}{dialect} \PYG{o}{=} \PYG{n}{csv}\PYG{o}{.}\PYG{n}{Sniffer}\PYG{p}{(}\PYG{p}{)}\PYG{o}{.}\PYG{n}{sniff}\PYG{p}{(}\PYG{n}{csvfile}\PYG{o}{.}\PYG{n}{read}\PYG{p}{(}\PYG{l+m+mi}{1024}\PYG{p}{)}\PYG{p}{)}
    \PYG{n}{csvfile}\PYG{o}{.}\PYG{n}{seek}\PYG{p}{(}\PYG{l+m+mi}{0}\PYG{p}{)}
    \PYG{n}{reader} \PYG{o}{=} \PYG{n}{csv}\PYG{o}{.}\PYG{n}{reader}\PYG{p}{(}\PYG{n}{csvfile}\PYG{p}{,} \PYG{n}{dialect}\PYG{p}{)}
    \PYG{c+c1}{\PYGZsh{} ... process CSV file contents here ...}
\end{sphinxVerbatim}

The {\hyperref[\detokenize{csv:module-csv}]{\sphinxcrossref{\sphinxcode{\sphinxupquote{csv}}}}} module defines the following constants:
\index{QUOTE\_ALL (in module csv)@\spxentry{QUOTE\_ALL}\spxextra{in module csv}}

\vspace{5px}

\begin{fulllineitems}
\phantomsection\stepcounter{subsection}
\addcontentsline{toc}{subsection}{\protect\numberline{\thesubsection}{QUOTE\_ALL}}
\phantomsection\label{\detokenize{csv:csv.QUOTE_ALL}}\pysigline{\sphinxcode{\sphinxupquote{csv.}}\sphinxbfcode{\sphinxupquote{QUOTE\_ALL}}}
Instructs {\hyperref[\detokenize{csv:csv.writer}]{\sphinxcrossref{\sphinxcode{\sphinxupquote{writer}}}}} objects to quote all fields.

\end{fulllineitems}

\index{QUOTE\_MINIMAL (in module csv)@\spxentry{QUOTE\_MINIMAL}\spxextra{in module csv}}

\vspace{5px}

\begin{fulllineitems}
\phantomsection\stepcounter{subsection}
\addcontentsline{toc}{subsection}{\protect\numberline{\thesubsection}{QUOTE\_MINIMAL}}
\phantomsection\label{\detokenize{csv:csv.QUOTE_MINIMAL}}\pysigline{\sphinxcode{\sphinxupquote{csv.}}\sphinxbfcode{\sphinxupquote{QUOTE\_MINIMAL}}}
Instructs {\hyperref[\detokenize{csv:csv.writer}]{\sphinxcrossref{\sphinxcode{\sphinxupquote{writer}}}}} objects to only quote those fields which contain
special characters such as \sphinxstyleemphasis{delimiter}, \sphinxstyleemphasis{quotechar} or any of the characters in
\sphinxstyleemphasis{lineterminator}.

\end{fulllineitems}

\index{QUOTE\_NONNUMERIC (in module csv)@\spxentry{QUOTE\_NONNUMERIC}\spxextra{in module csv}}

\vspace{5px}

\begin{fulllineitems}
\phantomsection\stepcounter{subsection}
\addcontentsline{toc}{subsection}{\protect\numberline{\thesubsection}{QUOTE\_NONNUMERIC}}
\phantomsection\label{\detokenize{csv:csv.QUOTE_NONNUMERIC}}\pysigline{\sphinxcode{\sphinxupquote{csv.}}\sphinxbfcode{\sphinxupquote{QUOTE\_NONNUMERIC}}}
Instructs {\hyperref[\detokenize{csv:csv.writer}]{\sphinxcrossref{\sphinxcode{\sphinxupquote{writer}}}}} objects to quote all non\sphinxhyphen{}numeric fields.

Instructs the reader to convert all non\sphinxhyphen{}quoted fields to type \sphinxstyleemphasis{float}.

\end{fulllineitems}

\index{QUOTE\_NONE (in module csv)@\spxentry{QUOTE\_NONE}\spxextra{in module csv}}

\vspace{5px}

\begin{fulllineitems}
\phantomsection\stepcounter{subsection}
\addcontentsline{toc}{subsection}{\protect\numberline{\thesubsection}{QUOTE\_NONE}}
\phantomsection\label{\detokenize{csv:csv.QUOTE_NONE}}\pysigline{\sphinxcode{\sphinxupquote{csv.}}\sphinxbfcode{\sphinxupquote{QUOTE\_NONE}}}
Instructs {\hyperref[\detokenize{csv:csv.writer}]{\sphinxcrossref{\sphinxcode{\sphinxupquote{writer}}}}} objects to never quote fields.  When the current
\sphinxstyleemphasis{delimiter} occurs in output data it is preceded by the current \sphinxstyleemphasis{escapechar}
character.  If \sphinxstyleemphasis{escapechar} is not set, the writer will raise {\hyperref[\detokenize{csv:csv.Error}]{\sphinxcrossref{\sphinxcode{\sphinxupquote{Error}}}}} if
any characters that require escaping are encountered.

Instructs {\hyperref[\detokenize{csv:csv.reader}]{\sphinxcrossref{\sphinxcode{\sphinxupquote{reader}}}}} to perform no special processing of quote characters.

\end{fulllineitems}


The {\hyperref[\detokenize{csv:module-csv}]{\sphinxcrossref{\sphinxcode{\sphinxupquote{csv}}}}} module defines the following exception:
\index{Error@\spxentry{Error}}

\vspace{5px}

\begin{fulllineitems}
\phantomsection\label{\detokenize{csv:csv.Error}}\pysigline{\sphinxbfcode{\sphinxupquote{exception }}\sphinxcode{\sphinxupquote{csv.}}\sphinxbfcode{\sphinxupquote{Error}}}
Raised by any of the functions when an error is detected.

\end{fulllineitems}



\section{Dialects and Formatting Parameters}
\label{\detokenize{csv:dialects-and-formatting-parameters}}\label{\detokenize{csv:csv-fmt-params}}
To make it easier to specify the format of input and output records, specific
formatting parameters are grouped together into dialects.  A dialect is a
subclass of the {\hyperref[\detokenize{csv:csv.Dialect}]{\sphinxcrossref{\sphinxcode{\sphinxupquote{Dialect}}}}} class having a set of specific methods and a
single \sphinxcode{\sphinxupquote{validate()}} method.  When creating {\hyperref[\detokenize{csv:csv.reader}]{\sphinxcrossref{\sphinxcode{\sphinxupquote{reader}}}}} or
{\hyperref[\detokenize{csv:csv.writer}]{\sphinxcrossref{\sphinxcode{\sphinxupquote{writer}}}}} objects, the programmer can specify a string or a subclass of
the {\hyperref[\detokenize{csv:csv.Dialect}]{\sphinxcrossref{\sphinxcode{\sphinxupquote{Dialect}}}}} class as the dialect parameter.  In addition to, or instead
of, the \sphinxstyleemphasis{dialect} parameter, the programmer can also specify individual
formatting parameters, which have the same names as the attributes defined below
for the {\hyperref[\detokenize{csv:csv.Dialect}]{\sphinxcrossref{\sphinxcode{\sphinxupquote{Dialect}}}}} class.

Dialects support the following attributes:
\index{delimiter (csv.Dialect attribute)@\spxentry{delimiter}\spxextra{csv.Dialect attribute}}

\vspace{5px}

\begin{fulllineitems}
\phantomsection\label{\detokenize{csv:csv.Dialect.delimiter}}\pysigline{\sphinxcode{\sphinxupquote{Dialect.}}\sphinxbfcode{\sphinxupquote{delimiter}}}
A one\sphinxhyphen{}character string used to separate fields.  It defaults to \sphinxcode{\sphinxupquote{\textquotesingle{},\textquotesingle{}}}.

\end{fulllineitems}

\index{doublequote (csv.Dialect attribute)@\spxentry{doublequote}\spxextra{csv.Dialect attribute}}

\vspace{5px}

\begin{fulllineitems}
\phantomsection\label{\detokenize{csv:csv.Dialect.doublequote}}\pysigline{\sphinxcode{\sphinxupquote{Dialect.}}\sphinxbfcode{\sphinxupquote{doublequote}}}
Controls how instances of \sphinxstyleemphasis{quotechar} appearing inside a field should
themselves be quoted.  When \sphinxcode{\sphinxupquote{True}}, the character is doubled. When
\sphinxcode{\sphinxupquote{False}}, the \sphinxstyleemphasis{escapechar} is used as a prefix to the \sphinxstyleemphasis{quotechar}.  It
defaults to \sphinxcode{\sphinxupquote{True}}.

On output, if \sphinxstyleemphasis{doublequote} is \sphinxcode{\sphinxupquote{False}} and no \sphinxstyleemphasis{escapechar} is set,
{\hyperref[\detokenize{csv:csv.Error}]{\sphinxcrossref{\sphinxcode{\sphinxupquote{Error}}}}} is raised if a \sphinxstyleemphasis{quotechar} is found in a field.

\end{fulllineitems}

\index{escapechar (csv.Dialect attribute)@\spxentry{escapechar}\spxextra{csv.Dialect attribute}}

\vspace{5px}

\begin{fulllineitems}
\phantomsection\label{\detokenize{csv:csv.Dialect.escapechar}}\pysigline{\sphinxcode{\sphinxupquote{Dialect.}}\sphinxbfcode{\sphinxupquote{escapechar}}}
A one\sphinxhyphen{}character string used by the writer to escape the \sphinxstyleemphasis{delimiter} if \sphinxstyleemphasis{quoting}
is set to {\hyperref[\detokenize{csv:csv.QUOTE_NONE}]{\sphinxcrossref{\sphinxcode{\sphinxupquote{QUOTE\_NONE}}}}} and the \sphinxstyleemphasis{quotechar} if \sphinxstyleemphasis{doublequote} is
\sphinxcode{\sphinxupquote{False}}. On reading, the \sphinxstyleemphasis{escapechar} removes any special meaning from
the following character. It defaults to \sphinxcode{\sphinxupquote{None}}, which disables escaping.

\end{fulllineitems}

\index{lineterminator (csv.Dialect attribute)@\spxentry{lineterminator}\spxextra{csv.Dialect attribute}}

\vspace{5px}

\begin{fulllineitems}
\phantomsection\label{\detokenize{csv:csv.Dialect.lineterminator}}\pysigline{\sphinxcode{\sphinxupquote{Dialect.}}\sphinxbfcode{\sphinxupquote{lineterminator}}}
The string used to terminate lines produced by the {\hyperref[\detokenize{csv:csv.writer}]{\sphinxcrossref{\sphinxcode{\sphinxupquote{writer}}}}}. It defaults
to \sphinxcode{\sphinxupquote{\textquotesingle{}\textbackslash{}r\textbackslash{}n\textquotesingle{}}}.

\begin{sphinxadmonition}{note}{Note:}
The {\hyperref[\detokenize{csv:csv.reader}]{\sphinxcrossref{\sphinxcode{\sphinxupquote{reader}}}}} is hard\sphinxhyphen{}coded to recognise either \sphinxcode{\sphinxupquote{\textquotesingle{}\textbackslash{}r\textquotesingle{}}} or \sphinxcode{\sphinxupquote{\textquotesingle{}\textbackslash{}n\textquotesingle{}}} as
end\sphinxhyphen{}of\sphinxhyphen{}line, and ignores \sphinxstyleemphasis{lineterminator}. This behavior may change in the
future.
\end{sphinxadmonition}

\end{fulllineitems}

\index{quotechar (csv.Dialect attribute)@\spxentry{quotechar}\spxextra{csv.Dialect attribute}}

\vspace{5px}

\begin{fulllineitems}
\phantomsection\label{\detokenize{csv:csv.Dialect.quotechar}}\pysigline{\sphinxcode{\sphinxupquote{Dialect.}}\sphinxbfcode{\sphinxupquote{quotechar}}}
A one\sphinxhyphen{}character string used to quote fields containing special characters, such
as the \sphinxstyleemphasis{delimiter} or \sphinxstyleemphasis{quotechar}, or which contain new\sphinxhyphen{}line characters.  It
defaults to \sphinxcode{\sphinxupquote{\textquotesingle{}"\textquotesingle{}}}.

\end{fulllineitems}

\index{quoting (csv.Dialect attribute)@\spxentry{quoting}\spxextra{csv.Dialect attribute}}

\vspace{5px}

\begin{fulllineitems}
\phantomsection\label{\detokenize{csv:csv.Dialect.quoting}}\pysigline{\sphinxcode{\sphinxupquote{Dialect.}}\sphinxbfcode{\sphinxupquote{quoting}}}
Controls when quotes should be generated by the writer and recognised by the
reader.  It can take on any of the \sphinxcode{\sphinxupquote{QUOTE\_*}} constants (see section
{\hyperref[\detokenize{csv:csv-contents}]{\sphinxcrossref{\DUrole{std,std-ref}{Module Contents}}}}) and defaults to {\hyperref[\detokenize{csv:csv.QUOTE_MINIMAL}]{\sphinxcrossref{\sphinxcode{\sphinxupquote{QUOTE\_MINIMAL}}}}}.

\end{fulllineitems}

\index{skipinitialspace (csv.Dialect attribute)@\spxentry{skipinitialspace}\spxextra{csv.Dialect attribute}}

\vspace{5px}

\begin{fulllineitems}
\phantomsection\label{\detokenize{csv:csv.Dialect.skipinitialspace}}\pysigline{\sphinxcode{\sphinxupquote{Dialect.}}\sphinxbfcode{\sphinxupquote{skipinitialspace}}}
When \sphinxcode{\sphinxupquote{True}}, whitespace immediately following the \sphinxstyleemphasis{delimiter} is ignored.
The default is \sphinxcode{\sphinxupquote{False}}.

\end{fulllineitems}

\index{strict (csv.Dialect attribute)@\spxentry{strict}\spxextra{csv.Dialect attribute}}

\vspace{5px}

\begin{fulllineitems}
\phantomsection\label{\detokenize{csv:csv.Dialect.strict}}\pysigline{\sphinxcode{\sphinxupquote{Dialect.}}\sphinxbfcode{\sphinxupquote{strict}}}
When \sphinxcode{\sphinxupquote{True}}, raise exception {\hyperref[\detokenize{csv:csv.Error}]{\sphinxcrossref{\sphinxcode{\sphinxupquote{Error}}}}} on bad CSV input.
The default is \sphinxcode{\sphinxupquote{False}}.

\end{fulllineitems}



\section{Reader Objects}
\label{\detokenize{csv:reader-objects}}
Reader objects ({\hyperref[\detokenize{csv:csv.DictReader}]{\sphinxcrossref{\sphinxcode{\sphinxupquote{DictReader}}}}} instances and objects returned by the
{\hyperref[\detokenize{csv:csv.reader}]{\sphinxcrossref{\sphinxcode{\sphinxupquote{reader()}}}}} function) have the following public methods:
\index{\_\_next\_\_() (csv.csvreader method)@\spxentry{\_\_next\_\_()}\spxextra{csv.csvreader method}}

\vspace{5px}

\begin{fulllineitems}
\phantomsection\stepcounter{subsubsection}
\addcontentsline{toc}{subsubsection}{\protect\numberline{\thesubsubsection}{\_\_next\_\_}}
\phantomsection\label{\detokenize{csv:csv.csvreader.__next__}}\pysiglinewithargsret{\sphinxcode{\sphinxupquote{csvreader.}}\sphinxbfcode{\sphinxupquote{\_\_next\_\_}}}{}{}
Return the next row of the reader’s iterable object as a list (if the object
was returned from {\hyperref[\detokenize{csv:csv.reader}]{\sphinxcrossref{\sphinxcode{\sphinxupquote{reader()}}}}}) or a dict (if it is a {\hyperref[\detokenize{csv:csv.DictReader}]{\sphinxcrossref{\sphinxcode{\sphinxupquote{DictReader}}}}}
instance), parsed according to the current dialect.  Usually you should call
this as \sphinxcode{\sphinxupquote{next(reader)}}.

\end{fulllineitems}


Reader objects have the following public attributes:
\index{dialect (csv.csvreader attribute)@\spxentry{dialect}\spxextra{csv.csvreader attribute}}

\vspace{5px}

\begin{fulllineitems}
\phantomsection\label{\detokenize{csv:csv.csvreader.dialect}}\pysigline{\sphinxcode{\sphinxupquote{csvreader.}}\sphinxbfcode{\sphinxupquote{dialect}}}
A read\sphinxhyphen{}only description of the dialect in use by the parser.

\end{fulllineitems}

\index{line\_num (csv.csvreader attribute)@\spxentry{line\_num}\spxextra{csv.csvreader attribute}}

\vspace{5px}

\begin{fulllineitems}
\phantomsection\label{\detokenize{csv:csv.csvreader.line_num}}\pysigline{\sphinxcode{\sphinxupquote{csvreader.}}\sphinxbfcode{\sphinxupquote{line\_num}}}
The number of lines read from the source iterator. This is not the same as the
number of records returned, as records can span multiple lines.

\end{fulllineitems}


DictReader objects have the following public attribute:
\index{fieldnames (csv.csvreader attribute)@\spxentry{fieldnames}\spxextra{csv.csvreader attribute}}

\vspace{5px}

\begin{fulllineitems}
\phantomsection\label{\detokenize{csv:csv.csvreader.fieldnames}}\pysigline{\sphinxcode{\sphinxupquote{csvreader.}}\sphinxbfcode{\sphinxupquote{fieldnames}}}
If not passed as a parameter when creating the object, this attribute is
initialized upon first access or when the first record is read from the
file.

\end{fulllineitems}



\section{Writer Objects}
\label{\detokenize{csv:writer-objects}}
\sphinxcode{\sphinxupquote{Writer}} objects ({\hyperref[\detokenize{csv:csv.DictWriter}]{\sphinxcrossref{\sphinxcode{\sphinxupquote{DictWriter}}}}} instances and objects returned by
the {\hyperref[\detokenize{csv:csv.writer}]{\sphinxcrossref{\sphinxcode{\sphinxupquote{writer()}}}}} function) have the following public methods.  A \sphinxstyleemphasis{row} must be
an iterable of strings or numbers for \sphinxcode{\sphinxupquote{Writer}} objects and a dictionary
mapping fieldnames to strings or numbers (by passing them through \sphinxcode{\sphinxupquote{str()}}
first) for {\hyperref[\detokenize{csv:csv.DictWriter}]{\sphinxcrossref{\sphinxcode{\sphinxupquote{DictWriter}}}}} objects.  Note that complex numbers are written
out surrounded by parens. This may cause some problems for other programs which
read CSV files (assuming they support complex numbers at all).
\index{writerow() (csv.csvwriter method)@\spxentry{writerow()}\spxextra{csv.csvwriter method}}

\vspace{5px}

\begin{fulllineitems}
\phantomsection\stepcounter{subsubsection}
\addcontentsline{toc}{subsubsection}{\protect\numberline{\thesubsubsection}{writerow}}
\phantomsection\label{\detokenize{csv:csv.csvwriter.writerow}}\pysiglinewithargsret{\sphinxcode{\sphinxupquote{csvwriter.}}\sphinxbfcode{\sphinxupquote{writerow}}}{\emph{\DUrole{n}{row}}}{}
Write the \sphinxstyleemphasis{row} parameter to the writer’s file object, formatted according to
the current dialect. Return the return value of the call to the \sphinxstyleemphasis{write} method
of the underlying file object.

\DUrole{versionmodified,changed}{Changed in version 3.5: }Added support of arbitrary iterables.

\end{fulllineitems}

\index{writerows() (csv.csvwriter method)@\spxentry{writerows()}\spxextra{csv.csvwriter method}}

\vspace{5px}

\begin{fulllineitems}
\phantomsection\stepcounter{subsubsection}
\addcontentsline{toc}{subsubsection}{\protect\numberline{\thesubsubsection}{writerows}}
\phantomsection\label{\detokenize{csv:csv.csvwriter.writerows}}\pysiglinewithargsret{\sphinxcode{\sphinxupquote{csvwriter.}}\sphinxbfcode{\sphinxupquote{writerows}}}{\emph{\DUrole{n}{rows}}}{}
Write all elements in \sphinxstyleemphasis{rows} (an iterable of \sphinxstyleemphasis{row} objects as described
above) to the writer’s file object, formatted according to the current
dialect.

\end{fulllineitems}


Writer objects have the following public attribute:
\index{dialect (csv.csvwriter attribute)@\spxentry{dialect}\spxextra{csv.csvwriter attribute}}

\vspace{5px}

\begin{fulllineitems}
\phantomsection\label{\detokenize{csv:csv.csvwriter.dialect}}\pysigline{\sphinxcode{\sphinxupquote{csvwriter.}}\sphinxbfcode{\sphinxupquote{dialect}}}
A read\sphinxhyphen{}only description of the dialect in use by the writer.

\end{fulllineitems}


DictWriter objects have the following public method:
\index{writeheader() (csv.DictWriter method)@\spxentry{writeheader()}\spxextra{csv.DictWriter method}}

\vspace{5px}

\begin{fulllineitems}
\phantomsection\stepcounter{subsubsection}
\addcontentsline{toc}{subsubsection}{\protect\numberline{\thesubsubsection}{writeheader}}
\phantomsection\label{\detokenize{csv:csv.DictWriter.writeheader}}\pysiglinewithargsret{\sphinxcode{\sphinxupquote{DictWriter.}}\sphinxbfcode{\sphinxupquote{writeheader}}}{}{}
Write a row with the field names (as specified in the constructor) to
the writer’s file object, formatted according to the current dialect. Return
the return value of the {\hyperref[\detokenize{csv:csv.csvwriter.writerow}]{\sphinxcrossref{\sphinxcode{\sphinxupquote{csvwriter.writerow()}}}}} call used internally.

\DUrole{versionmodified,added}{New in version 3.2.}

\DUrole{versionmodified,changed}{Changed in version 3.8: }{\hyperref[\detokenize{csv:csv.DictWriter.writeheader}]{\sphinxcrossref{\sphinxcode{\sphinxupquote{writeheader()}}}}} now also returns the value returned by
the {\hyperref[\detokenize{csv:csv.csvwriter.writerow}]{\sphinxcrossref{\sphinxcode{\sphinxupquote{csvwriter.writerow()}}}}} method it uses internally.

\end{fulllineitems}



\section{Examples}
\label{\detokenize{csv:examples}}\label{\detokenize{csv:csv-examples}}
The simplest example of reading a CSV file:

\begin{sphinxVerbatim}[commandchars=\\\{\}]
\PYG{k+kn}{import} \PYG{n+nn}{csv}
\PYG{k}{with} \PYG{n+nb}{open}\PYG{p}{(}\PYG{l+s+s1}{\PYGZsq{}}\PYG{l+s+s1}{some.csv}\PYG{l+s+s1}{\PYGZsq{}}\PYG{p}{,} \PYG{n}{newline}\PYG{o}{=}\PYG{l+s+s1}{\PYGZsq{}}\PYG{l+s+s1}{\PYGZsq{}}\PYG{p}{)} \PYG{k}{as} \PYG{n}{f}\PYG{p}{:}
    \PYG{n}{reader} \PYG{o}{=} \PYG{n}{csv}\PYG{o}{.}\PYG{n}{reader}\PYG{p}{(}\PYG{n}{f}\PYG{p}{)}
    \PYG{k}{for} \PYG{n}{row} \PYG{o+ow}{in} \PYG{n}{reader}\PYG{p}{:}
        \PYG{n+nb}{print}\PYG{p}{(}\PYG{n}{row}\PYG{p}{)}
\end{sphinxVerbatim}

Reading a file with an alternate format:

\begin{sphinxVerbatim}[commandchars=\\\{\}]
\PYG{k+kn}{import} \PYG{n+nn}{csv}
\PYG{k}{with} \PYG{n+nb}{open}\PYG{p}{(}\PYG{l+s+s1}{\PYGZsq{}}\PYG{l+s+s1}{passwd}\PYG{l+s+s1}{\PYGZsq{}}\PYG{p}{,} \PYG{n}{newline}\PYG{o}{=}\PYG{l+s+s1}{\PYGZsq{}}\PYG{l+s+s1}{\PYGZsq{}}\PYG{p}{)} \PYG{k}{as} \PYG{n}{f}\PYG{p}{:}
    \PYG{n}{reader} \PYG{o}{=} \PYG{n}{csv}\PYG{o}{.}\PYG{n}{reader}\PYG{p}{(}\PYG{n}{f}\PYG{p}{,} \PYG{n}{delimiter}\PYG{o}{=}\PYG{l+s+s1}{\PYGZsq{}}\PYG{l+s+s1}{:}\PYG{l+s+s1}{\PYGZsq{}}\PYG{p}{,} \PYG{n}{quoting}\PYG{o}{=}\PYG{n}{csv}\PYG{o}{.}\PYG{n}{QUOTE\PYGZus{}NONE}\PYG{p}{)}
    \PYG{k}{for} \PYG{n}{row} \PYG{o+ow}{in} \PYG{n}{reader}\PYG{p}{:}
        \PYG{n+nb}{print}\PYG{p}{(}\PYG{n}{row}\PYG{p}{)}
\end{sphinxVerbatim}

The corresponding simplest possible writing example is:

\begin{sphinxVerbatim}[commandchars=\\\{\}]
\PYG{k+kn}{import} \PYG{n+nn}{csv}
\PYG{k}{with} \PYG{n+nb}{open}\PYG{p}{(}\PYG{l+s+s1}{\PYGZsq{}}\PYG{l+s+s1}{some.csv}\PYG{l+s+s1}{\PYGZsq{}}\PYG{p}{,} \PYG{l+s+s1}{\PYGZsq{}}\PYG{l+s+s1}{w}\PYG{l+s+s1}{\PYGZsq{}}\PYG{p}{,} \PYG{n}{newline}\PYG{o}{=}\PYG{l+s+s1}{\PYGZsq{}}\PYG{l+s+s1}{\PYGZsq{}}\PYG{p}{)} \PYG{k}{as} \PYG{n}{f}\PYG{p}{:}
    \PYG{n}{writer} \PYG{o}{=} \PYG{n}{csv}\PYG{o}{.}\PYG{n}{writer}\PYG{p}{(}\PYG{n}{f}\PYG{p}{)}
    \PYG{n}{writer}\PYG{o}{.}\PYG{n}{writerows}\PYG{p}{(}\PYG{n}{someiterable}\PYG{p}{)}
\end{sphinxVerbatim}

Since \sphinxcode{\sphinxupquote{open()}} is used to open a CSV file for reading, the file
will by default be decoded into unicode using the system default
encoding (see \sphinxcode{\sphinxupquote{locale.getpreferredencoding()}}).  To decode a file
using a different encoding, use the \sphinxcode{\sphinxupquote{encoding}} argument of open:

\begin{sphinxVerbatim}[commandchars=\\\{\}]
\PYG{k+kn}{import} \PYG{n+nn}{csv}
\PYG{k}{with} \PYG{n+nb}{open}\PYG{p}{(}\PYG{l+s+s1}{\PYGZsq{}}\PYG{l+s+s1}{some.csv}\PYG{l+s+s1}{\PYGZsq{}}\PYG{p}{,} \PYG{n}{newline}\PYG{o}{=}\PYG{l+s+s1}{\PYGZsq{}}\PYG{l+s+s1}{\PYGZsq{}}\PYG{p}{,} \PYG{n}{encoding}\PYG{o}{=}\PYG{l+s+s1}{\PYGZsq{}}\PYG{l+s+s1}{utf\PYGZhy{}8}\PYG{l+s+s1}{\PYGZsq{}}\PYG{p}{)} \PYG{k}{as} \PYG{n}{f}\PYG{p}{:}
    \PYG{n}{reader} \PYG{o}{=} \PYG{n}{csv}\PYG{o}{.}\PYG{n}{reader}\PYG{p}{(}\PYG{n}{f}\PYG{p}{)}
    \PYG{k}{for} \PYG{n}{row} \PYG{o+ow}{in} \PYG{n}{reader}\PYG{p}{:}
        \PYG{n+nb}{print}\PYG{p}{(}\PYG{n}{row}\PYG{p}{)}
\end{sphinxVerbatim}

The same applies to writing in something other than the system default
encoding: specify the encoding argument when opening the output file.

Registering a new dialect:

\begin{sphinxVerbatim}[commandchars=\\\{\}]
\PYG{k+kn}{import} \PYG{n+nn}{csv}
\PYG{n}{csv}\PYG{o}{.}\PYG{n}{register\PYGZus{}dialect}\PYG{p}{(}\PYG{l+s+s1}{\PYGZsq{}}\PYG{l+s+s1}{unixpwd}\PYG{l+s+s1}{\PYGZsq{}}\PYG{p}{,} \PYG{n}{delimiter}\PYG{o}{=}\PYG{l+s+s1}{\PYGZsq{}}\PYG{l+s+s1}{:}\PYG{l+s+s1}{\PYGZsq{}}\PYG{p}{,} \PYG{n}{quoting}\PYG{o}{=}\PYG{n}{csv}\PYG{o}{.}\PYG{n}{QUOTE\PYGZus{}NONE}\PYG{p}{)}
\PYG{k}{with} \PYG{n+nb}{open}\PYG{p}{(}\PYG{l+s+s1}{\PYGZsq{}}\PYG{l+s+s1}{passwd}\PYG{l+s+s1}{\PYGZsq{}}\PYG{p}{,} \PYG{n}{newline}\PYG{o}{=}\PYG{l+s+s1}{\PYGZsq{}}\PYG{l+s+s1}{\PYGZsq{}}\PYG{p}{)} \PYG{k}{as} \PYG{n}{f}\PYG{p}{:}
    \PYG{n}{reader} \PYG{o}{=} \PYG{n}{csv}\PYG{o}{.}\PYG{n}{reader}\PYG{p}{(}\PYG{n}{f}\PYG{p}{,} \PYG{l+s+s1}{\PYGZsq{}}\PYG{l+s+s1}{unixpwd}\PYG{l+s+s1}{\PYGZsq{}}\PYG{p}{)}
\end{sphinxVerbatim}

A slightly more advanced use of the reader — catching and reporting errors:

\begin{sphinxVerbatim}[commandchars=\\\{\}]
\PYG{k+kn}{import} \PYG{n+nn}{csv}\PYG{o}{,} \PYG{n+nn}{sys}
\PYG{n}{filename} \PYG{o}{=} \PYG{l+s+s1}{\PYGZsq{}}\PYG{l+s+s1}{some.csv}\PYG{l+s+s1}{\PYGZsq{}}
\PYG{k}{with} \PYG{n+nb}{open}\PYG{p}{(}\PYG{n}{filename}\PYG{p}{,} \PYG{n}{newline}\PYG{o}{=}\PYG{l+s+s1}{\PYGZsq{}}\PYG{l+s+s1}{\PYGZsq{}}\PYG{p}{)} \PYG{k}{as} \PYG{n}{f}\PYG{p}{:}
    \PYG{n}{reader} \PYG{o}{=} \PYG{n}{csv}\PYG{o}{.}\PYG{n}{reader}\PYG{p}{(}\PYG{n}{f}\PYG{p}{)}
    \PYG{k}{try}\PYG{p}{:}
        \PYG{k}{for} \PYG{n}{row} \PYG{o+ow}{in} \PYG{n}{reader}\PYG{p}{:}
            \PYG{n+nb}{print}\PYG{p}{(}\PYG{n}{row}\PYG{p}{)}
    \PYG{k}{except} \PYG{n}{csv}\PYG{o}{.}\PYG{n}{Error} \PYG{k}{as} \PYG{n}{e}\PYG{p}{:}
        \PYG{n}{sys}\PYG{o}{.}\PYG{n}{exit}\PYG{p}{(}\PYG{l+s+s1}{\PYGZsq{}}\PYG{l+s+s1}{file }\PYG{l+s+si}{\PYGZob{}\PYGZcb{}}\PYG{l+s+s1}{, line }\PYG{l+s+si}{\PYGZob{}\PYGZcb{}}\PYG{l+s+s1}{: }\PYG{l+s+si}{\PYGZob{}\PYGZcb{}}\PYG{l+s+s1}{\PYGZsq{}}\PYG{o}{.}\PYG{n}{format}\PYG{p}{(}\PYG{n}{filename}\PYG{p}{,} \PYG{n}{reader}\PYG{o}{.}\PYG{n}{line\PYGZus{}num}\PYG{p}{,} \PYG{n}{e}\PYG{p}{)}\PYG{p}{)}
\end{sphinxVerbatim}

And while the module doesn’t directly support parsing strings, it can easily be
done:

\begin{sphinxVerbatim}[commandchars=\\\{\}]
\PYG{k+kn}{import} \PYG{n+nn}{csv}
\PYG{k}{for} \PYG{n}{row} \PYG{o+ow}{in} \PYG{n}{csv}\PYG{o}{.}\PYG{n}{reader}\PYG{p}{(}\PYG{p}{[}\PYG{l+s+s1}{\PYGZsq{}}\PYG{l+s+s1}{one,two,three}\PYG{l+s+s1}{\PYGZsq{}}\PYG{p}{]}\PYG{p}{)}\PYG{p}{:}
    \PYG{n+nb}{print}\PYG{p}{(}\PYG{n}{row}\PYG{p}{)}
\end{sphinxVerbatim}


\chapter{\sphinxstyleliteralintitle{\sphinxupquote{string}} — Common string operations}
\label{\detokenize{string:module-string}}\label{\detokenize{string:string-common-string-operations}}\label{\detokenize{string::doc}}\index{module@\spxentry{module}!string@\spxentry{string}}\index{string@\spxentry{string}!module@\spxentry{module}}

\section{String constants}
\label{\detokenize{string:string-constants}}
The constants defined in this module are:
\index{ascii\_letters (in module string)@\spxentry{ascii\_letters}\spxextra{in module string}}

\vspace{5px}

\begin{fulllineitems}
\phantomsection\stepcounter{subsection}
\addcontentsline{toc}{subsection}{\protect\numberline{\thesubsection}{ascii\_letters}}
\phantomsection\label{\detokenize{string:string.ascii_letters}}\pysigline{\sphinxcode{\sphinxupquote{string.}}\sphinxbfcode{\sphinxupquote{ascii\_letters}}}
The concatenation of the {\hyperref[\detokenize{string:string.ascii_lowercase}]{\sphinxcrossref{\sphinxcode{\sphinxupquote{ascii\_lowercase}}}}} and {\hyperref[\detokenize{string:string.ascii_uppercase}]{\sphinxcrossref{\sphinxcode{\sphinxupquote{ascii\_uppercase}}}}}
constants described below.  This value is not locale\sphinxhyphen{}dependent.

\end{fulllineitems}

\index{ascii\_lowercase (in module string)@\spxentry{ascii\_lowercase}\spxextra{in module string}}

\vspace{5px}

\begin{fulllineitems}
\phantomsection\stepcounter{subsection}
\addcontentsline{toc}{subsection}{\protect\numberline{\thesubsection}{ascii\_lowercase}}
\phantomsection\label{\detokenize{string:string.ascii_lowercase}}\pysigline{\sphinxcode{\sphinxupquote{string.}}\sphinxbfcode{\sphinxupquote{ascii\_lowercase}}}
The lowercase letters \sphinxcode{\sphinxupquote{\textquotesingle{}abcdefghijklmnopqrstuvwxyz\textquotesingle{}}}.  This value is not
locale\sphinxhyphen{}dependent and will not change.

\end{fulllineitems}

\index{ascii\_uppercase (in module string)@\spxentry{ascii\_uppercase}\spxextra{in module string}}

\vspace{5px}

\begin{fulllineitems}
\phantomsection\stepcounter{subsection}
\addcontentsline{toc}{subsection}{\protect\numberline{\thesubsection}{ascii\_uppercase}}
\phantomsection\label{\detokenize{string:string.ascii_uppercase}}\pysigline{\sphinxcode{\sphinxupquote{string.}}\sphinxbfcode{\sphinxupquote{ascii\_uppercase}}}
The uppercase letters \sphinxcode{\sphinxupquote{\textquotesingle{}ABCDEFGHIJKLMNOPQRSTUVWXYZ\textquotesingle{}}}.  This value is not
locale\sphinxhyphen{}dependent and will not change.

\end{fulllineitems}

\index{digits (in module string)@\spxentry{digits}\spxextra{in module string}}

\vspace{5px}

\begin{fulllineitems}
\phantomsection\stepcounter{subsection}
\addcontentsline{toc}{subsection}{\protect\numberline{\thesubsection}{digits}}
\phantomsection\label{\detokenize{string:string.digits}}\pysigline{\sphinxcode{\sphinxupquote{string.}}\sphinxbfcode{\sphinxupquote{digits}}}
The string \sphinxcode{\sphinxupquote{\textquotesingle{}0123456789\textquotesingle{}}}.

\end{fulllineitems}

\index{hexdigits (in module string)@\spxentry{hexdigits}\spxextra{in module string}}

\vspace{5px}

\begin{fulllineitems}
\phantomsection\stepcounter{subsection}
\addcontentsline{toc}{subsection}{\protect\numberline{\thesubsection}{hexdigits}}
\phantomsection\label{\detokenize{string:string.hexdigits}}\pysigline{\sphinxcode{\sphinxupquote{string.}}\sphinxbfcode{\sphinxupquote{hexdigits}}}
The string \sphinxcode{\sphinxupquote{\textquotesingle{}0123456789abcdefABCDEF\textquotesingle{}}}.

\end{fulllineitems}

\index{octdigits (in module string)@\spxentry{octdigits}\spxextra{in module string}}

\vspace{5px}

\begin{fulllineitems}
\phantomsection\stepcounter{subsection}
\addcontentsline{toc}{subsection}{\protect\numberline{\thesubsection}{octdigits}}
\phantomsection\label{\detokenize{string:string.octdigits}}\pysigline{\sphinxcode{\sphinxupquote{string.}}\sphinxbfcode{\sphinxupquote{octdigits}}}
The string \sphinxcode{\sphinxupquote{\textquotesingle{}01234567\textquotesingle{}}}.

\end{fulllineitems}

\index{punctuation (in module string)@\spxentry{punctuation}\spxextra{in module string}}

\vspace{5px}

\begin{fulllineitems}
\phantomsection\stepcounter{subsection}
\addcontentsline{toc}{subsection}{\protect\numberline{\thesubsection}{punctuation}}
\phantomsection\label{\detokenize{string:string.punctuation}}\pysigline{\sphinxcode{\sphinxupquote{string.}}\sphinxbfcode{\sphinxupquote{punctuation}}}
String of ASCII characters which are considered punctuation characters
in the \sphinxcode{\sphinxupquote{C}} locale: \sphinxcode{\sphinxupquote{!"\#\$\%\&\textquotesingle{}()*+,\sphinxhyphen{}./:;\textless{}=\textgreater{}?@{[}\textbackslash{}{]}\textasciicircum{}\_\textasciigrave{}\{|\}\textasciitilde{}}}.

\end{fulllineitems}

\index{printable (in module string)@\spxentry{printable}\spxextra{in module string}}

\vspace{5px}

\begin{fulllineitems}
\phantomsection\stepcounter{subsection}
\addcontentsline{toc}{subsection}{\protect\numberline{\thesubsection}{printable}}
\phantomsection\label{\detokenize{string:string.printable}}\pysigline{\sphinxcode{\sphinxupquote{string.}}\sphinxbfcode{\sphinxupquote{printable}}}
String of ASCII characters which are considered printable.  This is a
combination of {\hyperref[\detokenize{string:string.digits}]{\sphinxcrossref{\sphinxcode{\sphinxupquote{digits}}}}}, {\hyperref[\detokenize{string:string.ascii_letters}]{\sphinxcrossref{\sphinxcode{\sphinxupquote{ascii\_letters}}}}}, {\hyperref[\detokenize{string:string.punctuation}]{\sphinxcrossref{\sphinxcode{\sphinxupquote{punctuation}}}}},
and {\hyperref[\detokenize{string:string.whitespace}]{\sphinxcrossref{\sphinxcode{\sphinxupquote{whitespace}}}}}.

\end{fulllineitems}

\index{whitespace (in module string)@\spxentry{whitespace}\spxextra{in module string}}

\vspace{5px}

\begin{fulllineitems}
\phantomsection\stepcounter{subsection}
\addcontentsline{toc}{subsection}{\protect\numberline{\thesubsection}{whitespace}}
\phantomsection\label{\detokenize{string:string.whitespace}}\pysigline{\sphinxcode{\sphinxupquote{string.}}\sphinxbfcode{\sphinxupquote{whitespace}}}
A string containing all ASCII characters that are considered whitespace.
This includes the characters space, tab, linefeed, return, formfeed, and
vertical tab.

\end{fulllineitems}



\section{Custom String Formatting}
\label{\detokenize{string:custom-string-formatting}}\label{\detokenize{string:string-formatting}}
The built\sphinxhyphen{}in string class provides the ability to do complex variable
substitutions and value formatting via the \sphinxcode{\sphinxupquote{format()}} method described in
\index{Python Enhancement Proposals@\spxentry{Python Enhancement Proposals}!PEP 3101@\spxentry{PEP 3101}}\sphinxhref{https://www.python.org/dev/peps/pep-3101}{\sphinxstylestrong{PEP 3101}}.  The {\hyperref[\detokenize{string:string.Formatter}]{\sphinxcrossref{\sphinxcode{\sphinxupquote{Formatter}}}}} class in the {\hyperref[\detokenize{string:module-string}]{\sphinxcrossref{\sphinxcode{\sphinxupquote{string}}}}} module allows
you to create and customize your own string formatting behaviors using the same
implementation as the built\sphinxhyphen{}in \sphinxcode{\sphinxupquote{format()}} method.
\index{Formatter (class in string)@\spxentry{Formatter}\spxextra{class in string}}

\vspace{5px}

\begin{fulllineitems}
\phantomsection\stepcounter{subsection}
\addcontentsline{toc}{subsection}{\protect\numberline{\thesubsection}{Formatter}}
\phantomsection\label{\detokenize{string:string.Formatter}}\pysigline{\sphinxbfcode{\sphinxupquote{class }}\sphinxcode{\sphinxupquote{string.}}\sphinxbfcode{\sphinxupquote{Formatter}}}
The {\hyperref[\detokenize{string:string.Formatter}]{\sphinxcrossref{\sphinxcode{\sphinxupquote{Formatter}}}}} class has the following public methods:
\index{format() (string.Formatter method)@\spxentry{format()}\spxextra{string.Formatter method}}

\vspace{5px}

\begin{fulllineitems}
\phantomsection\stepcounter{subsubsection}
\addcontentsline{toc}{subsubsection}{\protect\numberline{\thesubsubsection}{format}}
\phantomsection\label{\detokenize{string:string.Formatter.format}}\pysiglinewithargsret{\sphinxbfcode{\sphinxupquote{format}}}{\emph{\DUrole{n}{format\_string}}, \emph{\DUrole{o}{/}}, \emph{\DUrole{o}{*}\DUrole{n}{args}}, \emph{\DUrole{o}{**}\DUrole{n}{kwargs}}}{}
The primary API method.  It takes a format string and
an arbitrary set of positional and keyword arguments.
It is just a wrapper that calls {\hyperref[\detokenize{string:string.Formatter.vformat}]{\sphinxcrossref{\sphinxcode{\sphinxupquote{vformat()}}}}}.

\DUrole{versionmodified,changed}{Changed in version 3.7: }A format string argument is now \DUrole{xref,std,std-ref}{positional\sphinxhyphen{}only}.

\end{fulllineitems}

\index{vformat() (string.Formatter method)@\spxentry{vformat()}\spxextra{string.Formatter method}}

\vspace{5px}

\begin{fulllineitems}
\phantomsection\stepcounter{subsubsection}
\addcontentsline{toc}{subsubsection}{\protect\numberline{\thesubsubsection}{vformat}}
\phantomsection\label{\detokenize{string:string.Formatter.vformat}}\pysiglinewithargsret{\sphinxbfcode{\sphinxupquote{vformat}}}{\emph{\DUrole{n}{format\_string}}, \emph{\DUrole{n}{args}}, \emph{\DUrole{n}{kwargs}}}{}
This function does the actual work of formatting.  It is exposed as a
separate function for cases where you want to pass in a predefined
dictionary of arguments, rather than unpacking and repacking the
dictionary as individual arguments using the \sphinxcode{\sphinxupquote{*args}} and \sphinxcode{\sphinxupquote{**kwargs}}
syntax.  {\hyperref[\detokenize{string:string.Formatter.vformat}]{\sphinxcrossref{\sphinxcode{\sphinxupquote{vformat()}}}}} does the work of breaking up the format string
into character data and replacement fields.  It calls the various
methods described below.

\end{fulllineitems}


In addition, the {\hyperref[\detokenize{string:string.Formatter}]{\sphinxcrossref{\sphinxcode{\sphinxupquote{Formatter}}}}} defines a number of methods that are
intended to be replaced by subclasses:
\index{parse() (string.Formatter method)@\spxentry{parse()}\spxextra{string.Formatter method}}

\vspace{5px}

\begin{fulllineitems}
\phantomsection\stepcounter{subsubsection}
\addcontentsline{toc}{subsubsection}{\protect\numberline{\thesubsubsection}{parse}}
\phantomsection\label{\detokenize{string:string.Formatter.parse}}\pysiglinewithargsret{\sphinxbfcode{\sphinxupquote{parse}}}{\emph{\DUrole{n}{format\_string}}}{}
Loop over the format\_string and return an iterable of tuples
(\sphinxstyleemphasis{literal\_text}, \sphinxstyleemphasis{field\_name}, \sphinxstyleemphasis{format\_spec}, \sphinxstyleemphasis{conversion}).  This is used
by {\hyperref[\detokenize{string:string.Formatter.vformat}]{\sphinxcrossref{\sphinxcode{\sphinxupquote{vformat()}}}}} to break the string into either literal text, or
replacement fields.

The values in the tuple conceptually represent a span of literal text
followed by a single replacement field.  If there is no literal text
(which can happen if two replacement fields occur consecutively), then
\sphinxstyleemphasis{literal\_text} will be a zero\sphinxhyphen{}length string.  If there is no replacement
field, then the values of \sphinxstyleemphasis{field\_name}, \sphinxstyleemphasis{format\_spec} and \sphinxstyleemphasis{conversion}
will be \sphinxcode{\sphinxupquote{None}}.

\end{fulllineitems}

\index{get\_field() (string.Formatter method)@\spxentry{get\_field()}\spxextra{string.Formatter method}}

\vspace{5px}

\begin{fulllineitems}
\phantomsection\stepcounter{subsubsection}
\addcontentsline{toc}{subsubsection}{\protect\numberline{\thesubsubsection}{get\_field}}
\phantomsection\label{\detokenize{string:string.Formatter.get_field}}\pysiglinewithargsret{\sphinxbfcode{\sphinxupquote{get\_field}}}{\emph{\DUrole{n}{field\_name}}, \emph{\DUrole{n}{args}}, \emph{\DUrole{n}{kwargs}}}{}
Given \sphinxstyleemphasis{field\_name} as returned by {\hyperref[\detokenize{string:string.Formatter.parse}]{\sphinxcrossref{\sphinxcode{\sphinxupquote{parse()}}}}} (see above), convert it to
an object to be formatted.  Returns a tuple (obj, used\_key).  The default
version takes strings of the form defined in \index{Python Enhancement Proposals@\spxentry{Python Enhancement Proposals}!PEP 3101@\spxentry{PEP 3101}}\sphinxhref{https://www.python.org/dev/peps/pep-3101}{\sphinxstylestrong{PEP 3101}}, such as
“0{[}name{]}” or “label.title”.  \sphinxstyleemphasis{args} and \sphinxstyleemphasis{kwargs} are as passed in to
{\hyperref[\detokenize{string:string.Formatter.vformat}]{\sphinxcrossref{\sphinxcode{\sphinxupquote{vformat()}}}}}.  The return value \sphinxstyleemphasis{used\_key} has the same meaning as the
\sphinxstyleemphasis{key} parameter to {\hyperref[\detokenize{string:string.Formatter.get_value}]{\sphinxcrossref{\sphinxcode{\sphinxupquote{get\_value()}}}}}.

\end{fulllineitems}

\index{get\_value() (string.Formatter method)@\spxentry{get\_value()}\spxextra{string.Formatter method}}

\vspace{5px}

\begin{fulllineitems}
\phantomsection\stepcounter{subsubsection}
\addcontentsline{toc}{subsubsection}{\protect\numberline{\thesubsubsection}{get\_value}}
\phantomsection\label{\detokenize{string:string.Formatter.get_value}}\pysiglinewithargsret{\sphinxbfcode{\sphinxupquote{get\_value}}}{\emph{\DUrole{n}{key}}, \emph{\DUrole{n}{args}}, \emph{\DUrole{n}{kwargs}}}{}
Retrieve a given field value.  The \sphinxstyleemphasis{key} argument will be either an
integer or a string.  If it is an integer, it represents the index of the
positional argument in \sphinxstyleemphasis{args}; if it is a string, then it represents a
named argument in \sphinxstyleemphasis{kwargs}.

The \sphinxstyleemphasis{args} parameter is set to the list of positional arguments to
{\hyperref[\detokenize{string:string.Formatter.vformat}]{\sphinxcrossref{\sphinxcode{\sphinxupquote{vformat()}}}}}, and the \sphinxstyleemphasis{kwargs} parameter is set to the dictionary of
keyword arguments.

For compound field names, these functions are only called for the first
component of the field name; subsequent components are handled through
normal attribute and indexing operations.

So for example, the field expression ‘0.name’ would cause
{\hyperref[\detokenize{string:string.Formatter.get_value}]{\sphinxcrossref{\sphinxcode{\sphinxupquote{get\_value()}}}}} to be called with a \sphinxstyleemphasis{key} argument of 0.  The \sphinxcode{\sphinxupquote{name}}
attribute will be looked up after {\hyperref[\detokenize{string:string.Formatter.get_value}]{\sphinxcrossref{\sphinxcode{\sphinxupquote{get\_value()}}}}} returns by calling the
built\sphinxhyphen{}in \sphinxcode{\sphinxupquote{getattr()}} function.

If the index or keyword refers to an item that does not exist, then an
\sphinxcode{\sphinxupquote{IndexError}} or \sphinxcode{\sphinxupquote{KeyError}} should be raised.

\end{fulllineitems}

\index{check\_unused\_args() (string.Formatter method)@\spxentry{check\_unused\_args()}\spxextra{string.Formatter method}}

\vspace{5px}

\begin{fulllineitems}
\phantomsection\stepcounter{subsubsection}
\addcontentsline{toc}{subsubsection}{\protect\numberline{\thesubsubsection}{check\_unused\_args}}
\phantomsection\label{\detokenize{string:string.Formatter.check_unused_args}}\pysiglinewithargsret{\sphinxbfcode{\sphinxupquote{check\_unused\_args}}}{\emph{\DUrole{n}{used\_args}}, \emph{\DUrole{n}{args}}, \emph{\DUrole{n}{kwargs}}}{}
Implement checking for unused arguments if desired.  The arguments to this
function is the set of all argument keys that were actually referred to in
the format string (integers for positional arguments, and strings for
named arguments), and a reference to the \sphinxstyleemphasis{args} and \sphinxstyleemphasis{kwargs} that was
passed to vformat.  The set of unused args can be calculated from these
parameters.  {\hyperref[\detokenize{string:string.Formatter.check_unused_args}]{\sphinxcrossref{\sphinxcode{\sphinxupquote{check\_unused\_args()}}}}} is assumed to raise an exception if
the check fails.

\end{fulllineitems}

\index{format\_field() (string.Formatter method)@\spxentry{format\_field()}\spxextra{string.Formatter method}}

\vspace{5px}

\begin{fulllineitems}
\phantomsection\stepcounter{subsubsection}
\addcontentsline{toc}{subsubsection}{\protect\numberline{\thesubsubsection}{format\_field}}
\phantomsection\label{\detokenize{string:string.Formatter.format_field}}\pysiglinewithargsret{\sphinxbfcode{\sphinxupquote{format\_field}}}{\emph{\DUrole{n}{value}}, \emph{\DUrole{n}{format\_spec}}}{}
{\hyperref[\detokenize{string:string.Formatter.format_field}]{\sphinxcrossref{\sphinxcode{\sphinxupquote{format\_field()}}}}} simply calls the global {\hyperref[\detokenize{string:string.Formatter.format}]{\sphinxcrossref{\sphinxcode{\sphinxupquote{format()}}}}} built\sphinxhyphen{}in.  The
method is provided so that subclasses can override it.

\end{fulllineitems}

\index{convert\_field() (string.Formatter method)@\spxentry{convert\_field()}\spxextra{string.Formatter method}}

\vspace{5px}

\begin{fulllineitems}
\phantomsection\stepcounter{subsubsection}
\addcontentsline{toc}{subsubsection}{\protect\numberline{\thesubsubsection}{convert\_field}}
\phantomsection\label{\detokenize{string:string.Formatter.convert_field}}\pysiglinewithargsret{\sphinxbfcode{\sphinxupquote{convert\_field}}}{\emph{\DUrole{n}{value}}, \emph{\DUrole{n}{conversion}}}{}
Converts the value (returned by {\hyperref[\detokenize{string:string.Formatter.get_field}]{\sphinxcrossref{\sphinxcode{\sphinxupquote{get\_field()}}}}}) given a conversion type
(as in the tuple returned by the {\hyperref[\detokenize{string:string.Formatter.parse}]{\sphinxcrossref{\sphinxcode{\sphinxupquote{parse()}}}}} method).  The default
version understands ‘s’ (str), ‘r’ (repr) and ‘a’ (ascii) conversion
types.

\end{fulllineitems}


\end{fulllineitems}



\section{Format String Syntax}
\label{\detokenize{string:format-string-syntax}}\label{\detokenize{string:formatstrings}}
The \sphinxcode{\sphinxupquote{str.format()}} method and the {\hyperref[\detokenize{string:string.Formatter}]{\sphinxcrossref{\sphinxcode{\sphinxupquote{Formatter}}}}} class share the same
syntax for format strings (although in the case of {\hyperref[\detokenize{string:string.Formatter}]{\sphinxcrossref{\sphinxcode{\sphinxupquote{Formatter}}}}},
subclasses can define their own format string syntax).  The syntax is
related to that of \DUrole{xref,std,std-ref}{formatted string literals}, but
there are differences.

\index{\sphinxleftcurlybrace{}\sphinxrightcurlybrace{} (curly brackets)@\spxentry{\sphinxleftcurlybrace{}\sphinxrightcurlybrace{}}\spxextra{curly brackets}!in string formatting@\spxentry{in string formatting}}\index{. (dot)@\spxentry{.}\spxextra{dot}!in string formatting@\spxentry{in string formatting}}\index{{[}{]} (square brackets)@\spxentry{{[}{]}}\spxextra{square brackets}!in string formatting@\spxentry{in string formatting}}\index{"! (exclamation)@\spxentry{"!}\spxextra{exclamation}!in string formatting@\spxentry{in string formatting}}\index{: (colon)@\spxentry{:}\spxextra{colon}!in string formatting@\spxentry{in string formatting}}\ignorespaces
Format strings contain “replacement fields” surrounded by curly braces \sphinxcode{\sphinxupquote{\{\}}}.
Anything that is not contained in braces is considered literal text, which is
copied unchanged to the output.  If you need to include a brace character in the
literal text, it can be escaped by doubling: \sphinxcode{\sphinxupquote{\{\{}} and \sphinxcode{\sphinxupquote{\}\}}}.

In less formal terms, the replacement field can start with a \sphinxstyleemphasis{field\_name} that specifies
the object whose value is to be formatted and inserted
into the output instead of the replacement field.
The \sphinxstyleemphasis{field\_name} is optionally followed by a  \sphinxstyleemphasis{conversion} field, which is
preceded by an exclamation point \sphinxcode{\sphinxupquote{\textquotesingle{}!\textquotesingle{}}}, and a \sphinxstyleemphasis{format\_spec}, which is preceded
by a colon \sphinxcode{\sphinxupquote{\textquotesingle{}:\textquotesingle{}}}.  These specify a non\sphinxhyphen{}default format for the replacement value.

See also the {\hyperref[\detokenize{string:formatspec}]{\sphinxcrossref{\DUrole{std,std-ref}{Format Specification Mini\sphinxhyphen{}Language}}}} section.

The \sphinxstyleemphasis{field\_name} itself begins with an \sphinxstyleemphasis{arg\_name} that is either a number or a
keyword.  If it’s a number, it refers to a positional argument, and if it’s a keyword,
it refers to a named keyword argument.  If the numerical arg\_names in a format string
are 0, 1, 2, … in sequence, they can all be omitted (not just some)
and the numbers 0, 1, 2, … will be automatically inserted in that order.
Because \sphinxstyleemphasis{arg\_name} is not quote\sphinxhyphen{}delimited, it is not possible to specify arbitrary
dictionary keys (e.g., the strings \sphinxcode{\sphinxupquote{\textquotesingle{}10\textquotesingle{}}} or \sphinxcode{\sphinxupquote{\textquotesingle{}:\sphinxhyphen{}{]}\textquotesingle{}}}) within a format string.
The \sphinxstyleemphasis{arg\_name} can be followed by any number of index or
attribute expressions. An expression of the form \sphinxcode{\sphinxupquote{\textquotesingle{}.name\textquotesingle{}}} selects the named
attribute using \sphinxcode{\sphinxupquote{getattr()}}, while an expression of the form \sphinxcode{\sphinxupquote{\textquotesingle{}{[}index{]}\textquotesingle{}}}
does an index lookup using \sphinxcode{\sphinxupquote{\_\_getitem\_\_()}}.

\DUrole{versionmodified,changed}{Changed in version 3.1: }The positional argument specifiers can be omitted for \sphinxcode{\sphinxupquote{str.format()}},
so \sphinxcode{\sphinxupquote{\textquotesingle{}\{\} \{\}\textquotesingle{}.format(a, b)}} is equivalent to \sphinxcode{\sphinxupquote{\textquotesingle{}\{0\} \{1\}\textquotesingle{}.format(a, b)}}.

\DUrole{versionmodified,changed}{Changed in version 3.4: }The positional argument specifiers can be omitted for {\hyperref[\detokenize{string:string.Formatter}]{\sphinxcrossref{\sphinxcode{\sphinxupquote{Formatter}}}}}.

Some simple format string examples:

\begin{sphinxVerbatim}[commandchars=\\\{\}]
\PYG{l+s+s2}{\PYGZdq{}}\PYG{l+s+s2}{First, thou shalt count to }\PYG{l+s+si}{\PYGZob{}0\PYGZcb{}}\PYG{l+s+s2}{\PYGZdq{}}  \PYG{c+c1}{\PYGZsh{} References first positional argument}
\PYG{l+s+s2}{\PYGZdq{}}\PYG{l+s+s2}{Bring me a }\PYG{l+s+si}{\PYGZob{}\PYGZcb{}}\PYG{l+s+s2}{\PYGZdq{}}                   \PYG{c+c1}{\PYGZsh{} Implicitly references the first positional argument}
\PYG{l+s+s2}{\PYGZdq{}}\PYG{l+s+s2}{From }\PYG{l+s+si}{\PYGZob{}\PYGZcb{}}\PYG{l+s+s2}{ to }\PYG{l+s+si}{\PYGZob{}\PYGZcb{}}\PYG{l+s+s2}{\PYGZdq{}}                   \PYG{c+c1}{\PYGZsh{} Same as \PYGZdq{}From \PYGZob{}0\PYGZcb{} to \PYGZob{}1\PYGZcb{}\PYGZdq{}}
\PYG{l+s+s2}{\PYGZdq{}}\PYG{l+s+s2}{My quest is }\PYG{l+s+si}{\PYGZob{}name\PYGZcb{}}\PYG{l+s+s2}{\PYGZdq{}}              \PYG{c+c1}{\PYGZsh{} References keyword argument \PYGZsq{}name\PYGZsq{}}
\PYG{l+s+s2}{\PYGZdq{}}\PYG{l+s+s2}{Weight in tons }\PYG{l+s+si}{\PYGZob{}0.weight\PYGZcb{}}\PYG{l+s+s2}{\PYGZdq{}}       \PYG{c+c1}{\PYGZsh{} \PYGZsq{}weight\PYGZsq{} attribute of first positional arg}
\PYG{l+s+s2}{\PYGZdq{}}\PYG{l+s+s2}{Units destroyed: }\PYG{l+s+si}{\PYGZob{}players[0]\PYGZcb{}}\PYG{l+s+s2}{\PYGZdq{}}   \PYG{c+c1}{\PYGZsh{} First element of keyword argument \PYGZsq{}players\PYGZsq{}.}
\end{sphinxVerbatim}

The \sphinxstyleemphasis{conversion} field causes a type coercion before formatting.  Normally, the
job of formatting a value is done by the \sphinxcode{\sphinxupquote{\_\_format\_\_()}} method of the value
itself.  However, in some cases it is desirable to force a type to be formatted
as a string, overriding its own definition of formatting.  By converting the
value to a string before calling \sphinxcode{\sphinxupquote{\_\_format\_\_()}}, the normal formatting logic
is bypassed.

Three conversion flags are currently supported: \sphinxcode{\sphinxupquote{\textquotesingle{}!s\textquotesingle{}}} which calls \sphinxcode{\sphinxupquote{str()}}
on the value, \sphinxcode{\sphinxupquote{\textquotesingle{}!r\textquotesingle{}}} which calls \sphinxcode{\sphinxupquote{repr()}} and \sphinxcode{\sphinxupquote{\textquotesingle{}!a\textquotesingle{}}} which calls
\sphinxcode{\sphinxupquote{ascii()}}.

Some examples:

\begin{sphinxVerbatim}[commandchars=\\\{\}]
\PYG{l+s+s2}{\PYGZdq{}}\PYG{l+s+s2}{Harold}\PYG{l+s+s2}{\PYGZsq{}}\PYG{l+s+s2}{s a clever }\PYG{l+s+si}{\PYGZob{}0!s\PYGZcb{}}\PYG{l+s+s2}{\PYGZdq{}}        \PYG{c+c1}{\PYGZsh{} Calls str() on the argument first}
\PYG{l+s+s2}{\PYGZdq{}}\PYG{l+s+s2}{Bring out the holy }\PYG{l+s+si}{\PYGZob{}name!r\PYGZcb{}}\PYG{l+s+s2}{\PYGZdq{}}    \PYG{c+c1}{\PYGZsh{} Calls repr() on the argument first}
\PYG{l+s+s2}{\PYGZdq{}}\PYG{l+s+s2}{More }\PYG{l+s+si}{\PYGZob{}!a\PYGZcb{}}\PYG{l+s+s2}{\PYGZdq{}}                      \PYG{c+c1}{\PYGZsh{} Calls ascii() on the argument first}
\end{sphinxVerbatim}

The \sphinxstyleemphasis{format\_spec} field contains a specification of how the value should be
presented, including such details as field width, alignment, padding, decimal
precision and so on.  Each value type can define its own “formatting
mini\sphinxhyphen{}language” or interpretation of the \sphinxstyleemphasis{format\_spec}.

Most built\sphinxhyphen{}in types support a common formatting mini\sphinxhyphen{}language, which is
described in the next section.

A \sphinxstyleemphasis{format\_spec} field can also include nested replacement fields within it.
These nested replacement fields may contain a field name, conversion flag
and format specification, but deeper nesting is
not allowed.  The replacement fields within the
format\_spec are substituted before the \sphinxstyleemphasis{format\_spec} string is interpreted.
This allows the formatting of a value to be dynamically specified.

See the {\hyperref[\detokenize{string:formatexamples}]{\sphinxcrossref{\DUrole{std,std-ref}{Format examples}}}} section for some examples.


\subsection{Format Specification Mini\sphinxhyphen{}Language}
\label{\detokenize{string:format-specification-mini-language}}\label{\detokenize{string:formatspec}}
“Format specifications” are used within replacement fields contained within a
format string to define how individual values are presented (see
{\hyperref[\detokenize{string:formatstrings}]{\sphinxcrossref{\DUrole{std,std-ref}{Format String Syntax}}}} and \DUrole{xref,std,std-ref}{f\sphinxhyphen{}strings}).
They can also be passed directly to the built\sphinxhyphen{}in
\sphinxcode{\sphinxupquote{format()}} function.  Each formattable type may define how the format
specification is to be interpreted.

Most built\sphinxhyphen{}in types implement the following options for format specifications,
although some of the formatting options are only supported by the numeric types.

A general convention is that an empty format specification produces
the same result as if you had called \sphinxcode{\sphinxupquote{str()}} on the value. A
non\sphinxhyphen{}empty format specification typically modifies the result.

If a valid \sphinxstyleemphasis{align} value is specified, it can be preceded by a \sphinxstyleemphasis{fill}
character that can be any character and defaults to a space if omitted.
It is not possible to use a literal curly brace (“\sphinxcode{\sphinxupquote{\{}}” or “\sphinxcode{\sphinxupquote{\}}}”) as
the \sphinxstyleemphasis{fill} character in a \DUrole{xref,std,std-ref}{formatted string literal} or when using the \sphinxcode{\sphinxupquote{str.format()}}
method.  However, it is possible to insert a curly brace
with a nested replacement field.  This limitation doesn’t
affect the \sphinxcode{\sphinxupquote{format()}} function.

The meaning of the various alignment options is as follows:
\begin{quote}

\index{\textless{} (less)@\spxentry{\textless{}}\spxextra{less}!in string formatting@\spxentry{in string formatting}}\index{\textgreater{} (greater)@\spxentry{\textgreater{}}\spxextra{greater}!in string formatting@\spxentry{in string formatting}}\index{= (equals)@\spxentry{=}\spxextra{equals}!in string formatting@\spxentry{in string formatting}}\index{\textasciicircum{} (caret)@\spxentry{\textasciicircum{}}\spxextra{caret}!in string formatting@\spxentry{in string formatting}}\ignorespaces

\begin{savenotes}\sphinxattablestart
\centering
\phantomsection\label{\detokenize{string:index-3}}\nobreak
\begin{tabulary}{\linewidth}[t]{|T|T|}
\hline
\sphinxstyletheadfamily
Option
&\sphinxstyletheadfamily
Meaning
\\
\hline
\sphinxcode{\sphinxupquote{\textquotesingle{}\textless{}\textquotesingle{}}}
&
Forces the field to be left\sphinxhyphen{}aligned within the available
space (this is the default for most objects).
\\
\hline
\sphinxcode{\sphinxupquote{\textquotesingle{}\textgreater{}\textquotesingle{}}}
&
Forces the field to be right\sphinxhyphen{}aligned within the
available space (this is the default for numbers).
\\
\hline
\sphinxcode{\sphinxupquote{\textquotesingle{}=\textquotesingle{}}}
&
Forces the padding to be placed after the sign (if any)
but before the digits.  This is used for printing fields
in the form ‘+000000120’. This alignment option is only
valid for numeric types.  It becomes the default when ‘0’
immediately precedes the field width.
\\
\hline
\sphinxcode{\sphinxupquote{\textquotesingle{}\textasciicircum{}\textquotesingle{}}}
&
Forces the field to be centered within the available
space.
\\
\hline
\end{tabulary}
\par
\sphinxattableend\end{savenotes}
\end{quote}

Note that unless a minimum field width is defined, the field width will always
be the same size as the data to fill it, so that the alignment option has no
meaning in this case.

The \sphinxstyleemphasis{sign} option is only valid for number types, and can be one of the
following:
\begin{quote}

\index{+ (plus)@\spxentry{+}\spxextra{plus}!in string formatting@\spxentry{in string formatting}}\index{\sphinxhyphen{} (minus)@\spxentry{\sphinxhyphen{}}\spxextra{minus}!in string formatting@\spxentry{in string formatting}}\index{space@\spxentry{space}!in string formatting@\spxentry{in string formatting}}\ignorespaces

\begin{savenotes}\sphinxattablestart
\centering
\phantomsection\label{\detokenize{string:index-4}}\nobreak
\begin{tabulary}{\linewidth}[t]{|T|T|}
\hline
\sphinxstyletheadfamily
Option
&\sphinxstyletheadfamily
Meaning
\\
\hline
\sphinxcode{\sphinxupquote{\textquotesingle{}+\textquotesingle{}}}
&
indicates that a sign should be used for both
positive as well as negative numbers.
\\
\hline
\sphinxcode{\sphinxupquote{\textquotesingle{}\sphinxhyphen{}\textquotesingle{}}}
&
indicates that a sign should be used only for negative
numbers (this is the default behavior).
\\
\hline
space
&
indicates that a leading space should be used on
positive numbers, and a minus sign on negative numbers.
\\
\hline
\end{tabulary}
\par
\sphinxattableend\end{savenotes}
\end{quote}

\index{\# (hash)@\spxentry{\#}\spxextra{hash}!in string formatting@\spxentry{in string formatting}}\ignorespaces
The \sphinxcode{\sphinxupquote{\textquotesingle{}\#\textquotesingle{}}} option causes the “alternate form” to be used for the
conversion.  The alternate form is defined differently for different
types.  This option is only valid for integer, float, complex and
Decimal types. For integers, when binary, octal, or hexadecimal output
is used, this option adds the prefix respective \sphinxcode{\sphinxupquote{\textquotesingle{}0b\textquotesingle{}}}, \sphinxcode{\sphinxupquote{\textquotesingle{}0o\textquotesingle{}}}, or
\sphinxcode{\sphinxupquote{\textquotesingle{}0x\textquotesingle{}}} to the output value. For floats, complex and Decimal the
alternate form causes the result of the conversion to always contain a
decimal\sphinxhyphen{}point character, even if no digits follow it. Normally, a
decimal\sphinxhyphen{}point character appears in the result of these conversions
only if a digit follows it. In addition, for \sphinxcode{\sphinxupquote{\textquotesingle{}g\textquotesingle{}}} and \sphinxcode{\sphinxupquote{\textquotesingle{}G\textquotesingle{}}}
conversions, trailing zeros are not removed from the result.

\index{, (comma)@\spxentry{,}\spxextra{comma}!in string formatting@\spxentry{in string formatting}}\ignorespaces
The \sphinxcode{\sphinxupquote{\textquotesingle{},\textquotesingle{}}} option signals the use of a comma for a thousands separator.
For a locale aware separator, use the \sphinxcode{\sphinxupquote{\textquotesingle{}n\textquotesingle{}}} integer presentation type
instead.

\DUrole{versionmodified,changed}{Changed in version 3.1: }Added the \sphinxcode{\sphinxupquote{\textquotesingle{},\textquotesingle{}}} option (see also \index{Python Enhancement Proposals@\spxentry{Python Enhancement Proposals}!PEP 378@\spxentry{PEP 378}}\sphinxhref{https://www.python.org/dev/peps/pep-0378}{\sphinxstylestrong{PEP 378}}).

\index{\_ (underscore)@\spxentry{\_}\spxextra{underscore}!in string formatting@\spxentry{in string formatting}}\ignorespaces
The \sphinxcode{\sphinxupquote{\textquotesingle{}\_\textquotesingle{}}} option signals the use of an underscore for a thousands
separator for floating point presentation types and for integer
presentation type \sphinxcode{\sphinxupquote{\textquotesingle{}d\textquotesingle{}}}.  For integer presentation types \sphinxcode{\sphinxupquote{\textquotesingle{}b\textquotesingle{}}},
\sphinxcode{\sphinxupquote{\textquotesingle{}o\textquotesingle{}}}, \sphinxcode{\sphinxupquote{\textquotesingle{}x\textquotesingle{}}}, and \sphinxcode{\sphinxupquote{\textquotesingle{}X\textquotesingle{}}}, underscores will be inserted every 4
digits.  For other presentation types, specifying this option is an
error.

\DUrole{versionmodified,changed}{Changed in version 3.6: }Added the \sphinxcode{\sphinxupquote{\textquotesingle{}\_\textquotesingle{}}} option (see also \index{Python Enhancement Proposals@\spxentry{Python Enhancement Proposals}!PEP 515@\spxentry{PEP 515}}\sphinxhref{https://www.python.org/dev/peps/pep-0515}{\sphinxstylestrong{PEP 515}}).

\sphinxstyleemphasis{width} is a decimal integer defining the minimum total field width,
including any prefixes, separators, and other formatting characters.
If not specified, then the field width will be determined by the content.

When no explicit alignment is given, preceding the \sphinxstyleemphasis{width} field by a zero
(\sphinxcode{\sphinxupquote{\textquotesingle{}0\textquotesingle{}}}) character enables
sign\sphinxhyphen{}aware zero\sphinxhyphen{}padding for numeric types.  This is equivalent to a \sphinxstyleemphasis{fill}
character of \sphinxcode{\sphinxupquote{\textquotesingle{}0\textquotesingle{}}} with an \sphinxstyleemphasis{alignment} type of \sphinxcode{\sphinxupquote{\textquotesingle{}=\textquotesingle{}}}.

The \sphinxstyleemphasis{precision} is a decimal number indicating how many digits should be
displayed after the decimal point for a floating point value formatted with
\sphinxcode{\sphinxupquote{\textquotesingle{}f\textquotesingle{}}} and \sphinxcode{\sphinxupquote{\textquotesingle{}F\textquotesingle{}}}, or before and after the decimal point for a floating point
value formatted with \sphinxcode{\sphinxupquote{\textquotesingle{}g\textquotesingle{}}} or \sphinxcode{\sphinxupquote{\textquotesingle{}G\textquotesingle{}}}.  For non\sphinxhyphen{}number types the field
indicates the maximum field size \sphinxhyphen{} in other words, how many characters will be
used from the field content. The \sphinxstyleemphasis{precision} is not allowed for integer values.

Finally, the \sphinxstyleemphasis{type} determines how the data should be presented.

The available string presentation types are:
\begin{quote}


\begin{savenotes}\sphinxattablestart
\centering
\begin{tabulary}{\linewidth}[t]{|T|T|}
\hline
\sphinxstyletheadfamily
Type
&\sphinxstyletheadfamily
Meaning
\\
\hline
\sphinxcode{\sphinxupquote{\textquotesingle{}s\textquotesingle{}}}
&
String format. This is the default type for strings and
may be omitted.
\\
\hline
None
&
The same as \sphinxcode{\sphinxupquote{\textquotesingle{}s\textquotesingle{}}}.
\\
\hline
\end{tabulary}
\par
\sphinxattableend\end{savenotes}
\end{quote}

The available integer presentation types are:
\begin{quote}


\begin{savenotes}\sphinxattablestart
\centering
\begin{tabulary}{\linewidth}[t]{|T|T|}
\hline
\sphinxstyletheadfamily
Type
&\sphinxstyletheadfamily
Meaning
\\
\hline
\sphinxcode{\sphinxupquote{\textquotesingle{}b\textquotesingle{}}}
&
Binary format. Outputs the number in base 2.
\\
\hline
\sphinxcode{\sphinxupquote{\textquotesingle{}c\textquotesingle{}}}
&
Character. Converts the integer to the corresponding
unicode character before printing.
\\
\hline
\sphinxcode{\sphinxupquote{\textquotesingle{}d\textquotesingle{}}}
&
Decimal Integer. Outputs the number in base 10.
\\
\hline
\sphinxcode{\sphinxupquote{\textquotesingle{}o\textquotesingle{}}}
&
Octal format. Outputs the number in base 8.
\\
\hline
\sphinxcode{\sphinxupquote{\textquotesingle{}x\textquotesingle{}}}
&
Hex format. Outputs the number in base 16, using
lower\sphinxhyphen{}case letters for the digits above 9.
\\
\hline
\sphinxcode{\sphinxupquote{\textquotesingle{}X\textquotesingle{}}}
&
Hex format. Outputs the number in base 16, using
upper\sphinxhyphen{}case letters for the digits above 9.
\\
\hline
\sphinxcode{\sphinxupquote{\textquotesingle{}n\textquotesingle{}}}
&
Number. This is the same as \sphinxcode{\sphinxupquote{\textquotesingle{}d\textquotesingle{}}}, except that it uses
the current locale setting to insert the appropriate
number separator characters.
\\
\hline
None
&
The same as \sphinxcode{\sphinxupquote{\textquotesingle{}d\textquotesingle{}}}.
\\
\hline
\end{tabulary}
\par
\sphinxattableend\end{savenotes}
\end{quote}

In addition to the above presentation types, integers can be formatted
with the floating point presentation types listed below (except
\sphinxcode{\sphinxupquote{\textquotesingle{}n\textquotesingle{}}} and \sphinxcode{\sphinxupquote{None}}). When doing so, \sphinxcode{\sphinxupquote{float()}} is used to convert the
integer to a floating point number before formatting.

The available presentation types for floating point and decimal values are:
\begin{quote}


\begin{savenotes}\sphinxattablestart
\centering
\begin{tabulary}{\linewidth}[t]{|T|T|}
\hline
\sphinxstyletheadfamily
Type
&\sphinxstyletheadfamily
Meaning
\\
\hline
\sphinxcode{\sphinxupquote{\textquotesingle{}e\textquotesingle{}}}
&
Exponent notation. Prints the number in scientific
notation using the letter ‘e’ to indicate the exponent.
The default precision is \sphinxcode{\sphinxupquote{6}}.
\\
\hline
\sphinxcode{\sphinxupquote{\textquotesingle{}E\textquotesingle{}}}
&
Exponent notation. Same as \sphinxcode{\sphinxupquote{\textquotesingle{}e\textquotesingle{}}} except it uses an
upper case ‘E’ as the separator character.
\\
\hline
\sphinxcode{\sphinxupquote{\textquotesingle{}f\textquotesingle{}}}
&
Fixed\sphinxhyphen{}point notation. Displays the number as a
fixed\sphinxhyphen{}point number.  The default precision is \sphinxcode{\sphinxupquote{6}}.
\\
\hline
\sphinxcode{\sphinxupquote{\textquotesingle{}F\textquotesingle{}}}
&
Fixed\sphinxhyphen{}point notation. Same as \sphinxcode{\sphinxupquote{\textquotesingle{}f\textquotesingle{}}}, but converts
\sphinxcode{\sphinxupquote{nan}} to  \sphinxcode{\sphinxupquote{NAN}} and \sphinxcode{\sphinxupquote{inf}} to \sphinxcode{\sphinxupquote{INF}}.
\\
\hline
\sphinxcode{\sphinxupquote{\textquotesingle{}g\textquotesingle{}}}
&
General format.  For a given precision \sphinxcode{\sphinxupquote{p \textgreater{}= 1}},
this rounds the number to \sphinxcode{\sphinxupquote{p}} significant digits and
then formats the result in either fixed\sphinxhyphen{}point format
or in scientific notation, depending on its magnitude.

The precise rules are as follows: suppose that the
result formatted with presentation type \sphinxcode{\sphinxupquote{\textquotesingle{}e\textquotesingle{}}} and
precision \sphinxcode{\sphinxupquote{p\sphinxhyphen{}1}} would have exponent \sphinxcode{\sphinxupquote{exp}}.  Then,
if \sphinxcode{\sphinxupquote{m \textless{}= exp \textless{} p}}, where \sphinxcode{\sphinxupquote{m}} is \sphinxhyphen{}4 for floats and \sphinxhyphen{}6
for \sphinxcode{\sphinxupquote{Decimals}}, the number is
formatted with presentation type \sphinxcode{\sphinxupquote{\textquotesingle{}f\textquotesingle{}}} and precision
\sphinxcode{\sphinxupquote{p\sphinxhyphen{}1\sphinxhyphen{}exp}}.  Otherwise, the number is formatted
with presentation type \sphinxcode{\sphinxupquote{\textquotesingle{}e\textquotesingle{}}} and precision \sphinxcode{\sphinxupquote{p\sphinxhyphen{}1}}.
In both cases insignificant trailing zeros are removed
from the significand, and the decimal point is also
removed if there are no remaining digits following it,
unless the \sphinxcode{\sphinxupquote{\textquotesingle{}\#\textquotesingle{}}} option is used.

Positive and negative infinity, positive and negative
zero, and nans, are formatted as \sphinxcode{\sphinxupquote{inf}}, \sphinxcode{\sphinxupquote{\sphinxhyphen{}inf}},
\sphinxcode{\sphinxupquote{0}}, \sphinxcode{\sphinxupquote{\sphinxhyphen{}0}} and \sphinxcode{\sphinxupquote{nan}} respectively, regardless of
the precision.

A precision of \sphinxcode{\sphinxupquote{0}} is treated as equivalent to a
precision of \sphinxcode{\sphinxupquote{1}}.  The default precision is \sphinxcode{\sphinxupquote{6}}.
\\
\hline
\sphinxcode{\sphinxupquote{\textquotesingle{}G\textquotesingle{}}}
&
General format. Same as \sphinxcode{\sphinxupquote{\textquotesingle{}g\textquotesingle{}}} except switches to
\sphinxcode{\sphinxupquote{\textquotesingle{}E\textquotesingle{}}} if the number gets too large. The
representations of infinity and NaN are uppercased, too.
\\
\hline
\sphinxcode{\sphinxupquote{\textquotesingle{}n\textquotesingle{}}}
&
Number. This is the same as \sphinxcode{\sphinxupquote{\textquotesingle{}g\textquotesingle{}}}, except that it uses
the current locale setting to insert the appropriate
number separator characters.
\\
\hline
\sphinxcode{\sphinxupquote{\textquotesingle{}\%\textquotesingle{}}}
&
Percentage. Multiplies the number by 100 and displays
in fixed (\sphinxcode{\sphinxupquote{\textquotesingle{}f\textquotesingle{}}}) format, followed by a percent sign.
\\
\hline
None
&
Similar to \sphinxcode{\sphinxupquote{\textquotesingle{}g\textquotesingle{}}}, except that fixed\sphinxhyphen{}point notation,
when used, has at least one digit past the decimal point.
The default precision is as high as needed to represent
the particular value. The overall effect is to match the
output of \sphinxcode{\sphinxupquote{str()}} as altered by the other format
modifiers.
\\
\hline
\end{tabulary}
\par
\sphinxattableend\end{savenotes}
\end{quote}


\subsection{Format examples}
\label{\detokenize{string:format-examples}}\label{\detokenize{string:formatexamples}}
This section contains examples of the \sphinxcode{\sphinxupquote{str.format()}} syntax and
comparison with the old \sphinxcode{\sphinxupquote{\%}}\sphinxhyphen{}formatting.

In most of the cases the syntax is similar to the old \sphinxcode{\sphinxupquote{\%}}\sphinxhyphen{}formatting, with the
addition of the \sphinxcode{\sphinxupquote{\{\}}} and with \sphinxcode{\sphinxupquote{:}} used instead of \sphinxcode{\sphinxupquote{\%}}.
For example, \sphinxcode{\sphinxupquote{\textquotesingle{}\%03.2f\textquotesingle{}}} can be translated to \sphinxcode{\sphinxupquote{\textquotesingle{}\{:03.2f\}\textquotesingle{}}}.

The new format syntax also supports new and different options, shown in the
following examples.

Accessing arguments by position:

\begin{sphinxVerbatim}[commandchars=\\\{\}]
\PYG{g+gp}{\PYGZgt{}\PYGZgt{}\PYGZgt{} }\PYG{l+s+s1}{\PYGZsq{}}\PYG{l+s+si}{\PYGZob{}0\PYGZcb{}}\PYG{l+s+s1}{, }\PYG{l+s+si}{\PYGZob{}1\PYGZcb{}}\PYG{l+s+s1}{, }\PYG{l+s+si}{\PYGZob{}2\PYGZcb{}}\PYG{l+s+s1}{\PYGZsq{}}\PYG{o}{.}\PYG{n}{format}\PYG{p}{(}\PYG{l+s+s1}{\PYGZsq{}}\PYG{l+s+s1}{a}\PYG{l+s+s1}{\PYGZsq{}}\PYG{p}{,} \PYG{l+s+s1}{\PYGZsq{}}\PYG{l+s+s1}{b}\PYG{l+s+s1}{\PYGZsq{}}\PYG{p}{,} \PYG{l+s+s1}{\PYGZsq{}}\PYG{l+s+s1}{c}\PYG{l+s+s1}{\PYGZsq{}}\PYG{p}{)}
\PYG{g+go}{\PYGZsq{}a, b, c\PYGZsq{}}
\PYG{g+gp}{\PYGZgt{}\PYGZgt{}\PYGZgt{} }\PYG{l+s+s1}{\PYGZsq{}}\PYG{l+s+si}{\PYGZob{}\PYGZcb{}}\PYG{l+s+s1}{, }\PYG{l+s+si}{\PYGZob{}\PYGZcb{}}\PYG{l+s+s1}{, }\PYG{l+s+si}{\PYGZob{}\PYGZcb{}}\PYG{l+s+s1}{\PYGZsq{}}\PYG{o}{.}\PYG{n}{format}\PYG{p}{(}\PYG{l+s+s1}{\PYGZsq{}}\PYG{l+s+s1}{a}\PYG{l+s+s1}{\PYGZsq{}}\PYG{p}{,} \PYG{l+s+s1}{\PYGZsq{}}\PYG{l+s+s1}{b}\PYG{l+s+s1}{\PYGZsq{}}\PYG{p}{,} \PYG{l+s+s1}{\PYGZsq{}}\PYG{l+s+s1}{c}\PYG{l+s+s1}{\PYGZsq{}}\PYG{p}{)}  \PYG{c+c1}{\PYGZsh{} 3.1+ only}
\PYG{g+go}{\PYGZsq{}a, b, c\PYGZsq{}}
\PYG{g+gp}{\PYGZgt{}\PYGZgt{}\PYGZgt{} }\PYG{l+s+s1}{\PYGZsq{}}\PYG{l+s+si}{\PYGZob{}2\PYGZcb{}}\PYG{l+s+s1}{, }\PYG{l+s+si}{\PYGZob{}1\PYGZcb{}}\PYG{l+s+s1}{, }\PYG{l+s+si}{\PYGZob{}0\PYGZcb{}}\PYG{l+s+s1}{\PYGZsq{}}\PYG{o}{.}\PYG{n}{format}\PYG{p}{(}\PYG{l+s+s1}{\PYGZsq{}}\PYG{l+s+s1}{a}\PYG{l+s+s1}{\PYGZsq{}}\PYG{p}{,} \PYG{l+s+s1}{\PYGZsq{}}\PYG{l+s+s1}{b}\PYG{l+s+s1}{\PYGZsq{}}\PYG{p}{,} \PYG{l+s+s1}{\PYGZsq{}}\PYG{l+s+s1}{c}\PYG{l+s+s1}{\PYGZsq{}}\PYG{p}{)}
\PYG{g+go}{\PYGZsq{}c, b, a\PYGZsq{}}
\PYG{g+gp}{\PYGZgt{}\PYGZgt{}\PYGZgt{} }\PYG{l+s+s1}{\PYGZsq{}}\PYG{l+s+si}{\PYGZob{}2\PYGZcb{}}\PYG{l+s+s1}{, }\PYG{l+s+si}{\PYGZob{}1\PYGZcb{}}\PYG{l+s+s1}{, }\PYG{l+s+si}{\PYGZob{}0\PYGZcb{}}\PYG{l+s+s1}{\PYGZsq{}}\PYG{o}{.}\PYG{n}{format}\PYG{p}{(}\PYG{o}{*}\PYG{l+s+s1}{\PYGZsq{}}\PYG{l+s+s1}{abc}\PYG{l+s+s1}{\PYGZsq{}}\PYG{p}{)}      \PYG{c+c1}{\PYGZsh{} unpacking argument sequence}
\PYG{g+go}{\PYGZsq{}c, b, a\PYGZsq{}}
\PYG{g+gp}{\PYGZgt{}\PYGZgt{}\PYGZgt{} }\PYG{l+s+s1}{\PYGZsq{}}\PYG{l+s+si}{\PYGZob{}0\PYGZcb{}}\PYG{l+s+si}{\PYGZob{}1\PYGZcb{}}\PYG{l+s+si}{\PYGZob{}0\PYGZcb{}}\PYG{l+s+s1}{\PYGZsq{}}\PYG{o}{.}\PYG{n}{format}\PYG{p}{(}\PYG{l+s+s1}{\PYGZsq{}}\PYG{l+s+s1}{abra}\PYG{l+s+s1}{\PYGZsq{}}\PYG{p}{,} \PYG{l+s+s1}{\PYGZsq{}}\PYG{l+s+s1}{cad}\PYG{l+s+s1}{\PYGZsq{}}\PYG{p}{)}   \PYG{c+c1}{\PYGZsh{} arguments\PYGZsq{} indices can be repeated}
\PYG{g+go}{\PYGZsq{}abracadabra\PYGZsq{}}
\end{sphinxVerbatim}

Accessing arguments by name:

\begin{sphinxVerbatim}[commandchars=\\\{\}]
\PYG{g+gp}{\PYGZgt{}\PYGZgt{}\PYGZgt{} }\PYG{l+s+s1}{\PYGZsq{}}\PYG{l+s+s1}{Coordinates: }\PYG{l+s+si}{\PYGZob{}latitude\PYGZcb{}}\PYG{l+s+s1}{, }\PYG{l+s+si}{\PYGZob{}longitude\PYGZcb{}}\PYG{l+s+s1}{\PYGZsq{}}\PYG{o}{.}\PYG{n}{format}\PYG{p}{(}\PYG{n}{latitude}\PYG{o}{=}\PYG{l+s+s1}{\PYGZsq{}}\PYG{l+s+s1}{37.24N}\PYG{l+s+s1}{\PYGZsq{}}\PYG{p}{,} \PYG{n}{longitude}\PYG{o}{=}\PYG{l+s+s1}{\PYGZsq{}}\PYG{l+s+s1}{\PYGZhy{}115.81W}\PYG{l+s+s1}{\PYGZsq{}}\PYG{p}{)}
\PYG{g+go}{\PYGZsq{}Coordinates: 37.24N, \PYGZhy{}115.81W\PYGZsq{}}
\PYG{g+gp}{\PYGZgt{}\PYGZgt{}\PYGZgt{} }\PYG{n}{coord} \PYG{o}{=} \PYG{p}{\PYGZob{}}\PYG{l+s+s1}{\PYGZsq{}}\PYG{l+s+s1}{latitude}\PYG{l+s+s1}{\PYGZsq{}}\PYG{p}{:} \PYG{l+s+s1}{\PYGZsq{}}\PYG{l+s+s1}{37.24N}\PYG{l+s+s1}{\PYGZsq{}}\PYG{p}{,} \PYG{l+s+s1}{\PYGZsq{}}\PYG{l+s+s1}{longitude}\PYG{l+s+s1}{\PYGZsq{}}\PYG{p}{:} \PYG{l+s+s1}{\PYGZsq{}}\PYG{l+s+s1}{\PYGZhy{}115.81W}\PYG{l+s+s1}{\PYGZsq{}}\PYG{p}{\PYGZcb{}}
\PYG{g+gp}{\PYGZgt{}\PYGZgt{}\PYGZgt{} }\PYG{l+s+s1}{\PYGZsq{}}\PYG{l+s+s1}{Coordinates: }\PYG{l+s+si}{\PYGZob{}latitude\PYGZcb{}}\PYG{l+s+s1}{, }\PYG{l+s+si}{\PYGZob{}longitude\PYGZcb{}}\PYG{l+s+s1}{\PYGZsq{}}\PYG{o}{.}\PYG{n}{format}\PYG{p}{(}\PYG{o}{*}\PYG{o}{*}\PYG{n}{coord}\PYG{p}{)}
\PYG{g+go}{\PYGZsq{}Coordinates: 37.24N, \PYGZhy{}115.81W\PYGZsq{}}
\end{sphinxVerbatim}

Accessing arguments’ attributes:

\begin{sphinxVerbatim}[commandchars=\\\{\}]
\PYG{g+gp}{\PYGZgt{}\PYGZgt{}\PYGZgt{} }\PYG{n}{c} \PYG{o}{=} \PYG{l+m+mi}{3}\PYG{o}{\PYGZhy{}}\PYG{l+m+mi}{5}\PYG{n}{j}
\PYG{g+gp}{\PYGZgt{}\PYGZgt{}\PYGZgt{} }\PYG{p}{(}\PYG{l+s+s1}{\PYGZsq{}}\PYG{l+s+s1}{The complex number }\PYG{l+s+si}{\PYGZob{}0\PYGZcb{}}\PYG{l+s+s1}{ is formed from the real part }\PYG{l+s+si}{\PYGZob{}0.real\PYGZcb{}}\PYG{l+s+s1}{ }\PYG{l+s+s1}{\PYGZsq{}}
\PYG{g+gp}{... } \PYG{l+s+s1}{\PYGZsq{}}\PYG{l+s+s1}{and the imaginary part }\PYG{l+s+si}{\PYGZob{}0.imag\PYGZcb{}}\PYG{l+s+s1}{.}\PYG{l+s+s1}{\PYGZsq{}}\PYG{p}{)}\PYG{o}{.}\PYG{n}{format}\PYG{p}{(}\PYG{n}{c}\PYG{p}{)}
\PYG{g+go}{\PYGZsq{}The complex number (3\PYGZhy{}5j) is formed from the real part 3.0 and the imaginary part \PYGZhy{}5.0.\PYGZsq{}}
\PYG{g+gp}{\PYGZgt{}\PYGZgt{}\PYGZgt{} }\PYG{k}{class} \PYG{n+nc}{Point}\PYG{p}{:}
\PYG{g+gp}{... }    \PYG{k}{def} \PYG{n+nf+fm}{\PYGZus{}\PYGZus{}init\PYGZus{}\PYGZus{}}\PYG{p}{(}\PYG{n+nb+bp}{self}\PYG{p}{,} \PYG{n}{x}\PYG{p}{,} \PYG{n}{y}\PYG{p}{)}\PYG{p}{:}
\PYG{g+gp}{... }        \PYG{n+nb+bp}{self}\PYG{o}{.}\PYG{n}{x}\PYG{p}{,} \PYG{n+nb+bp}{self}\PYG{o}{.}\PYG{n}{y} \PYG{o}{=} \PYG{n}{x}\PYG{p}{,} \PYG{n}{y}
\PYG{g+gp}{... }    \PYG{k}{def} \PYG{n+nf+fm}{\PYGZus{}\PYGZus{}str\PYGZus{}\PYGZus{}}\PYG{p}{(}\PYG{n+nb+bp}{self}\PYG{p}{)}\PYG{p}{:}
\PYG{g+gp}{... }        \PYG{k}{return} \PYG{l+s+s1}{\PYGZsq{}}\PYG{l+s+s1}{Point(}\PYG{l+s+si}{\PYGZob{}self.x\PYGZcb{}}\PYG{l+s+s1}{, }\PYG{l+s+si}{\PYGZob{}self.y\PYGZcb{}}\PYG{l+s+s1}{)}\PYG{l+s+s1}{\PYGZsq{}}\PYG{o}{.}\PYG{n}{format}\PYG{p}{(}\PYG{n+nb+bp}{self}\PYG{o}{=}\PYG{n+nb+bp}{self}\PYG{p}{)}
\PYG{g+gp}{...}
\PYG{g+gp}{\PYGZgt{}\PYGZgt{}\PYGZgt{} }\PYG{n+nb}{str}\PYG{p}{(}\PYG{n}{Point}\PYG{p}{(}\PYG{l+m+mi}{4}\PYG{p}{,} \PYG{l+m+mi}{2}\PYG{p}{)}\PYG{p}{)}
\PYG{g+go}{\PYGZsq{}Point(4, 2)\PYGZsq{}}
\end{sphinxVerbatim}

Accessing arguments’ items:

\begin{sphinxVerbatim}[commandchars=\\\{\}]
\PYG{g+gp}{\PYGZgt{}\PYGZgt{}\PYGZgt{} }\PYG{n}{coord} \PYG{o}{=} \PYG{p}{(}\PYG{l+m+mi}{3}\PYG{p}{,} \PYG{l+m+mi}{5}\PYG{p}{)}
\PYG{g+gp}{\PYGZgt{}\PYGZgt{}\PYGZgt{} }\PYG{l+s+s1}{\PYGZsq{}}\PYG{l+s+s1}{X: }\PYG{l+s+si}{\PYGZob{}0[0]\PYGZcb{}}\PYG{l+s+s1}{;  Y: }\PYG{l+s+si}{\PYGZob{}0[1]\PYGZcb{}}\PYG{l+s+s1}{\PYGZsq{}}\PYG{o}{.}\PYG{n}{format}\PYG{p}{(}\PYG{n}{coord}\PYG{p}{)}
\PYG{g+go}{\PYGZsq{}X: 3;  Y: 5\PYGZsq{}}
\end{sphinxVerbatim}

Replacing \sphinxcode{\sphinxupquote{\%s}} and \sphinxcode{\sphinxupquote{\%r}}:

\begin{sphinxVerbatim}[commandchars=\\\{\}]
\PYG{g+gp}{\PYGZgt{}\PYGZgt{}\PYGZgt{} }\PYG{l+s+s2}{\PYGZdq{}}\PYG{l+s+s2}{repr() shows quotes: }\PYG{l+s+si}{\PYGZob{}!r\PYGZcb{}}\PYG{l+s+s2}{; str() doesn}\PYG{l+s+s2}{\PYGZsq{}}\PYG{l+s+s2}{t: }\PYG{l+s+si}{\PYGZob{}!s\PYGZcb{}}\PYG{l+s+s2}{\PYGZdq{}}\PYG{o}{.}\PYG{n}{format}\PYG{p}{(}\PYG{l+s+s1}{\PYGZsq{}}\PYG{l+s+s1}{test1}\PYG{l+s+s1}{\PYGZsq{}}\PYG{p}{,} \PYG{l+s+s1}{\PYGZsq{}}\PYG{l+s+s1}{test2}\PYG{l+s+s1}{\PYGZsq{}}\PYG{p}{)}
\PYG{g+go}{\PYGZdq{}repr() shows quotes: \PYGZsq{}test1\PYGZsq{}; str() doesn\PYGZsq{}t: test2\PYGZdq{}}
\end{sphinxVerbatim}

Aligning the text and specifying a width:

\begin{sphinxVerbatim}[commandchars=\\\{\}]
\PYG{g+gp}{\PYGZgt{}\PYGZgt{}\PYGZgt{} }\PYG{l+s+s1}{\PYGZsq{}}\PYG{l+s+si}{\PYGZob{}:\PYGZlt{}30\PYGZcb{}}\PYG{l+s+s1}{\PYGZsq{}}\PYG{o}{.}\PYG{n}{format}\PYG{p}{(}\PYG{l+s+s1}{\PYGZsq{}}\PYG{l+s+s1}{left aligned}\PYG{l+s+s1}{\PYGZsq{}}\PYG{p}{)}
\PYG{g+go}{\PYGZsq{}left aligned                  \PYGZsq{}}
\PYG{g+gp}{\PYGZgt{}\PYGZgt{}\PYGZgt{} }\PYG{l+s+s1}{\PYGZsq{}}\PYG{l+s+si}{\PYGZob{}:\PYGZgt{}30\PYGZcb{}}\PYG{l+s+s1}{\PYGZsq{}}\PYG{o}{.}\PYG{n}{format}\PYG{p}{(}\PYG{l+s+s1}{\PYGZsq{}}\PYG{l+s+s1}{right aligned}\PYG{l+s+s1}{\PYGZsq{}}\PYG{p}{)}
\PYG{g+go}{\PYGZsq{}                 right aligned\PYGZsq{}}
\PYG{g+gp}{\PYGZgt{}\PYGZgt{}\PYGZgt{} }\PYG{l+s+s1}{\PYGZsq{}}\PYG{l+s+si}{\PYGZob{}:\PYGZca{}30\PYGZcb{}}\PYG{l+s+s1}{\PYGZsq{}}\PYG{o}{.}\PYG{n}{format}\PYG{p}{(}\PYG{l+s+s1}{\PYGZsq{}}\PYG{l+s+s1}{centered}\PYG{l+s+s1}{\PYGZsq{}}\PYG{p}{)}
\PYG{g+go}{\PYGZsq{}           centered           \PYGZsq{}}
\PYG{g+gp}{\PYGZgt{}\PYGZgt{}\PYGZgt{} }\PYG{l+s+s1}{\PYGZsq{}}\PYG{l+s+si}{\PYGZob{}:*\PYGZca{}30\PYGZcb{}}\PYG{l+s+s1}{\PYGZsq{}}\PYG{o}{.}\PYG{n}{format}\PYG{p}{(}\PYG{l+s+s1}{\PYGZsq{}}\PYG{l+s+s1}{centered}\PYG{l+s+s1}{\PYGZsq{}}\PYG{p}{)}  \PYG{c+c1}{\PYGZsh{} use \PYGZsq{}*\PYGZsq{} as a fill char}
\PYG{g+go}{\PYGZsq{}***********centered***********\PYGZsq{}}
\end{sphinxVerbatim}

Replacing \sphinxcode{\sphinxupquote{\%+f}}, \sphinxcode{\sphinxupquote{\%\sphinxhyphen{}f}}, and \sphinxcode{\sphinxupquote{\% f}} and specifying a sign:

\begin{sphinxVerbatim}[commandchars=\\\{\}]
\PYG{g+gp}{\PYGZgt{}\PYGZgt{}\PYGZgt{} }\PYG{l+s+s1}{\PYGZsq{}}\PYG{l+s+si}{\PYGZob{}:+f\PYGZcb{}}\PYG{l+s+s1}{; }\PYG{l+s+si}{\PYGZob{}:+f\PYGZcb{}}\PYG{l+s+s1}{\PYGZsq{}}\PYG{o}{.}\PYG{n}{format}\PYG{p}{(}\PYG{l+m+mf}{3.14}\PYG{p}{,} \PYG{o}{\PYGZhy{}}\PYG{l+m+mf}{3.14}\PYG{p}{)}  \PYG{c+c1}{\PYGZsh{} show it always}
\PYG{g+go}{\PYGZsq{}+3.140000; \PYGZhy{}3.140000\PYGZsq{}}
\PYG{g+gp}{\PYGZgt{}\PYGZgt{}\PYGZgt{} }\PYG{l+s+s1}{\PYGZsq{}}\PYG{l+s+si}{\PYGZob{}: f\PYGZcb{}}\PYG{l+s+s1}{; }\PYG{l+s+si}{\PYGZob{}: f\PYGZcb{}}\PYG{l+s+s1}{\PYGZsq{}}\PYG{o}{.}\PYG{n}{format}\PYG{p}{(}\PYG{l+m+mf}{3.14}\PYG{p}{,} \PYG{o}{\PYGZhy{}}\PYG{l+m+mf}{3.14}\PYG{p}{)}  \PYG{c+c1}{\PYGZsh{} show a space for positive numbers}
\PYG{g+go}{\PYGZsq{} 3.140000; \PYGZhy{}3.140000\PYGZsq{}}
\PYG{g+gp}{\PYGZgt{}\PYGZgt{}\PYGZgt{} }\PYG{l+s+s1}{\PYGZsq{}}\PYG{l+s+si}{\PYGZob{}:\PYGZhy{}f\PYGZcb{}}\PYG{l+s+s1}{; }\PYG{l+s+si}{\PYGZob{}:\PYGZhy{}f\PYGZcb{}}\PYG{l+s+s1}{\PYGZsq{}}\PYG{o}{.}\PYG{n}{format}\PYG{p}{(}\PYG{l+m+mf}{3.14}\PYG{p}{,} \PYG{o}{\PYGZhy{}}\PYG{l+m+mf}{3.14}\PYG{p}{)}  \PYG{c+c1}{\PYGZsh{} show only the minus \PYGZhy{}\PYGZhy{} same as \PYGZsq{}\PYGZob{}:f\PYGZcb{}; \PYGZob{}:f\PYGZcb{}\PYGZsq{}}
\PYG{g+go}{\PYGZsq{}3.140000; \PYGZhy{}3.140000\PYGZsq{}}
\end{sphinxVerbatim}

Replacing \sphinxcode{\sphinxupquote{\%x}} and \sphinxcode{\sphinxupquote{\%o}} and converting the value to different bases:

\begin{sphinxVerbatim}[commandchars=\\\{\}]
\PYG{g+gp}{\PYGZgt{}\PYGZgt{}\PYGZgt{} }\PYG{c+c1}{\PYGZsh{} format also supports binary numbers}
\PYG{g+gp}{\PYGZgt{}\PYGZgt{}\PYGZgt{} }\PYG{l+s+s2}{\PYGZdq{}}\PYG{l+s+s2}{int: }\PYG{l+s+si}{\PYGZob{}0:d\PYGZcb{}}\PYG{l+s+s2}{;  hex: }\PYG{l+s+si}{\PYGZob{}0:x\PYGZcb{}}\PYG{l+s+s2}{;  oct: }\PYG{l+s+si}{\PYGZob{}0:o\PYGZcb{}}\PYG{l+s+s2}{;  bin: }\PYG{l+s+si}{\PYGZob{}0:b\PYGZcb{}}\PYG{l+s+s2}{\PYGZdq{}}\PYG{o}{.}\PYG{n}{format}\PYG{p}{(}\PYG{l+m+mi}{42}\PYG{p}{)}
\PYG{g+go}{\PYGZsq{}int: 42;  hex: 2a;  oct: 52;  bin: 101010\PYGZsq{}}
\PYG{g+gp}{\PYGZgt{}\PYGZgt{}\PYGZgt{} }\PYG{c+c1}{\PYGZsh{} with 0x, 0o, or 0b as prefix:}
\PYG{g+gp}{\PYGZgt{}\PYGZgt{}\PYGZgt{} }\PYG{l+s+s2}{\PYGZdq{}}\PYG{l+s+s2}{int: }\PYG{l+s+si}{\PYGZob{}0:d\PYGZcb{}}\PYG{l+s+s2}{;  hex: }\PYG{l+s+si}{\PYGZob{}0:\PYGZsh{}x\PYGZcb{}}\PYG{l+s+s2}{;  oct: }\PYG{l+s+si}{\PYGZob{}0:\PYGZsh{}o\PYGZcb{}}\PYG{l+s+s2}{;  bin: }\PYG{l+s+si}{\PYGZob{}0:\PYGZsh{}b\PYGZcb{}}\PYG{l+s+s2}{\PYGZdq{}}\PYG{o}{.}\PYG{n}{format}\PYG{p}{(}\PYG{l+m+mi}{42}\PYG{p}{)}
\PYG{g+go}{\PYGZsq{}int: 42;  hex: 0x2a;  oct: 0o52;  bin: 0b101010\PYGZsq{}}
\end{sphinxVerbatim}

Using the comma as a thousands separator:

\begin{sphinxVerbatim}[commandchars=\\\{\}]
\PYG{g+gp}{\PYGZgt{}\PYGZgt{}\PYGZgt{} }\PYG{l+s+s1}{\PYGZsq{}}\PYG{l+s+si}{\PYGZob{}:,\PYGZcb{}}\PYG{l+s+s1}{\PYGZsq{}}\PYG{o}{.}\PYG{n}{format}\PYG{p}{(}\PYG{l+m+mi}{1234567890}\PYG{p}{)}
\PYG{g+go}{\PYGZsq{}1,234,567,890\PYGZsq{}}
\end{sphinxVerbatim}

Expressing a percentage:

\begin{sphinxVerbatim}[commandchars=\\\{\}]
\PYG{g+gp}{\PYGZgt{}\PYGZgt{}\PYGZgt{} }\PYG{n}{points} \PYG{o}{=} \PYG{l+m+mi}{19}
\PYG{g+gp}{\PYGZgt{}\PYGZgt{}\PYGZgt{} }\PYG{n}{total} \PYG{o}{=} \PYG{l+m+mi}{22}
\PYG{g+gp}{\PYGZgt{}\PYGZgt{}\PYGZgt{} }\PYG{l+s+s1}{\PYGZsq{}}\PYG{l+s+s1}{Correct answers: }\PYG{l+s+si}{\PYGZob{}:.2\PYGZpc{}\PYGZcb{}}\PYG{l+s+s1}{\PYGZsq{}}\PYG{o}{.}\PYG{n}{format}\PYG{p}{(}\PYG{n}{points}\PYG{o}{/}\PYG{n}{total}\PYG{p}{)}
\PYG{g+go}{\PYGZsq{}Correct answers: 86.36\PYGZpc{}\PYGZsq{}}
\end{sphinxVerbatim}

Using type\sphinxhyphen{}specific formatting:

\begin{sphinxVerbatim}[commandchars=\\\{\}]
\PYG{g+gp}{\PYGZgt{}\PYGZgt{}\PYGZgt{} }\PYG{k+kn}{import} \PYG{n+nn}{datetime}
\PYG{g+gp}{\PYGZgt{}\PYGZgt{}\PYGZgt{} }\PYG{n}{d} \PYG{o}{=} \PYG{n}{datetime}\PYG{o}{.}\PYG{n}{datetime}\PYG{p}{(}\PYG{l+m+mi}{2010}\PYG{p}{,} \PYG{l+m+mi}{7}\PYG{p}{,} \PYG{l+m+mi}{4}\PYG{p}{,} \PYG{l+m+mi}{12}\PYG{p}{,} \PYG{l+m+mi}{15}\PYG{p}{,} \PYG{l+m+mi}{58}\PYG{p}{)}
\PYG{g+gp}{\PYGZgt{}\PYGZgt{}\PYGZgt{} }\PYG{l+s+s1}{\PYGZsq{}}\PYG{l+s+s1}{\PYGZob{}}\PYG{l+s+s1}{:}\PYG{l+s+s1}{\PYGZpc{}}\PYG{l+s+s1}{Y\PYGZhy{}}\PYG{l+s+s1}{\PYGZpc{}}\PYG{l+s+s1}{m\PYGZhy{}}\PYG{l+s+si}{\PYGZpc{}d}\PYG{l+s+s1}{ }\PYG{l+s+s1}{\PYGZpc{}}\PYG{l+s+s1}{H:}\PYG{l+s+s1}{\PYGZpc{}}\PYG{l+s+s1}{M:}\PYG{l+s+s1}{\PYGZpc{}}\PYG{l+s+s1}{S\PYGZcb{}}\PYG{l+s+s1}{\PYGZsq{}}\PYG{o}{.}\PYG{n}{format}\PYG{p}{(}\PYG{n}{d}\PYG{p}{)}
\PYG{g+go}{\PYGZsq{}2010\PYGZhy{}07\PYGZhy{}04 12:15:58\PYGZsq{}}
\end{sphinxVerbatim}

Nesting arguments and more complex examples:

\begin{sphinxVerbatim}[commandchars=\\\{\}]
\PYG{g+gp}{\PYGZgt{}\PYGZgt{}\PYGZgt{} }\PYG{k}{for} \PYG{n}{align}\PYG{p}{,} \PYG{n}{text} \PYG{o+ow}{in} \PYG{n+nb}{zip}\PYG{p}{(}\PYG{l+s+s1}{\PYGZsq{}}\PYG{l+s+s1}{\PYGZlt{}\PYGZca{}\PYGZgt{}}\PYG{l+s+s1}{\PYGZsq{}}\PYG{p}{,} \PYG{p}{[}\PYG{l+s+s1}{\PYGZsq{}}\PYG{l+s+s1}{left}\PYG{l+s+s1}{\PYGZsq{}}\PYG{p}{,} \PYG{l+s+s1}{\PYGZsq{}}\PYG{l+s+s1}{center}\PYG{l+s+s1}{\PYGZsq{}}\PYG{p}{,} \PYG{l+s+s1}{\PYGZsq{}}\PYG{l+s+s1}{right}\PYG{l+s+s1}{\PYGZsq{}}\PYG{p}{]}\PYG{p}{)}\PYG{p}{:}
\PYG{g+gp}{... }    \PYG{l+s+s1}{\PYGZsq{}}\PYG{l+s+s1}{\PYGZob{}}\PYG{l+s+s1}{0:}\PYG{l+s+si}{\PYGZob{}fill\PYGZcb{}}\PYG{l+s+si}{\PYGZob{}align\PYGZcb{}}\PYG{l+s+s1}{16\PYGZcb{}}\PYG{l+s+s1}{\PYGZsq{}}\PYG{o}{.}\PYG{n}{format}\PYG{p}{(}\PYG{n}{text}\PYG{p}{,} \PYG{n}{fill}\PYG{o}{=}\PYG{n}{align}\PYG{p}{,} \PYG{n}{align}\PYG{o}{=}\PYG{n}{align}\PYG{p}{)}
\PYG{g+gp}{...}
\PYG{g+go}{\PYGZsq{}left\PYGZlt{}\PYGZlt{}\PYGZlt{}\PYGZlt{}\PYGZlt{}\PYGZlt{}\PYGZlt{}\PYGZlt{}\PYGZlt{}\PYGZlt{}\PYGZlt{}\PYGZlt{}\PYGZsq{}}
\PYG{g+go}{\PYGZsq{}\PYGZca{}\PYGZca{}\PYGZca{}\PYGZca{}\PYGZca{}center\PYGZca{}\PYGZca{}\PYGZca{}\PYGZca{}\PYGZca{}\PYGZsq{}}
\PYG{g+go}{\PYGZsq{}\PYGZgt{}\PYGZgt{}\PYGZgt{}\PYGZgt{}\PYGZgt{}\PYGZgt{}\PYGZgt{}\PYGZgt{}\PYGZgt{}\PYGZgt{}\PYGZgt{}right\PYGZsq{}}
\PYG{g+go}{\PYGZgt{}\PYGZgt{}\PYGZgt{}}
\PYG{g+gp}{\PYGZgt{}\PYGZgt{}\PYGZgt{} }\PYG{n}{octets} \PYG{o}{=} \PYG{p}{[}\PYG{l+m+mi}{192}\PYG{p}{,} \PYG{l+m+mi}{168}\PYG{p}{,} \PYG{l+m+mi}{0}\PYG{p}{,} \PYG{l+m+mi}{1}\PYG{p}{]}
\PYG{g+gp}{\PYGZgt{}\PYGZgt{}\PYGZgt{} }\PYG{l+s+s1}{\PYGZsq{}}\PYG{l+s+si}{\PYGZob{}:02X\PYGZcb{}}\PYG{l+s+si}{\PYGZob{}:02X\PYGZcb{}}\PYG{l+s+si}{\PYGZob{}:02X\PYGZcb{}}\PYG{l+s+si}{\PYGZob{}:02X\PYGZcb{}}\PYG{l+s+s1}{\PYGZsq{}}\PYG{o}{.}\PYG{n}{format}\PYG{p}{(}\PYG{o}{*}\PYG{n}{octets}\PYG{p}{)}
\PYG{g+go}{\PYGZsq{}C0A80001\PYGZsq{}}
\PYG{g+gp}{\PYGZgt{}\PYGZgt{}\PYGZgt{} }\PYG{n+nb}{int}\PYG{p}{(}\PYG{n}{\PYGZus{}}\PYG{p}{,} \PYG{l+m+mi}{16}\PYG{p}{)}
\PYG{g+go}{3232235521}
\PYG{g+go}{\PYGZgt{}\PYGZgt{}\PYGZgt{}}
\PYG{g+gp}{\PYGZgt{}\PYGZgt{}\PYGZgt{} }\PYG{n}{width} \PYG{o}{=} \PYG{l+m+mi}{5}
\PYG{g+gp}{\PYGZgt{}\PYGZgt{}\PYGZgt{} }\PYG{k}{for} \PYG{n}{num} \PYG{o+ow}{in} \PYG{n+nb}{range}\PYG{p}{(}\PYG{l+m+mi}{5}\PYG{p}{,}\PYG{l+m+mi}{12}\PYG{p}{)}\PYG{p}{:}
\PYG{g+gp}{... }    \PYG{k}{for} \PYG{n}{base} \PYG{o+ow}{in} \PYG{l+s+s1}{\PYGZsq{}}\PYG{l+s+s1}{dXob}\PYG{l+s+s1}{\PYGZsq{}}\PYG{p}{:}
\PYG{g+gp}{... }        \PYG{n+nb}{print}\PYG{p}{(}\PYG{l+s+s1}{\PYGZsq{}}\PYG{l+s+s1}{\PYGZob{}}\PYG{l+s+s1}{0:}\PYG{l+s+si}{\PYGZob{}width\PYGZcb{}}\PYG{l+s+si}{\PYGZob{}base\PYGZcb{}}\PYG{l+s+s1}{\PYGZcb{}}\PYG{l+s+s1}{\PYGZsq{}}\PYG{o}{.}\PYG{n}{format}\PYG{p}{(}\PYG{n}{num}\PYG{p}{,} \PYG{n}{base}\PYG{o}{=}\PYG{n}{base}\PYG{p}{,} \PYG{n}{width}\PYG{o}{=}\PYG{n}{width}\PYG{p}{)}\PYG{p}{,} \PYG{n}{end}\PYG{o}{=}\PYG{l+s+s1}{\PYGZsq{}}\PYG{l+s+s1}{ }\PYG{l+s+s1}{\PYGZsq{}}\PYG{p}{)}
\PYG{g+gp}{... }    \PYG{n+nb}{print}\PYG{p}{(}\PYG{p}{)}
\PYG{g+gp}{...}
\PYG{g+go}{    5     5     5   101}
\PYG{g+go}{    6     6     6   110}
\PYG{g+go}{    7     7     7   111}
\PYG{g+go}{    8     8    10  1000}
\PYG{g+go}{    9     9    11  1001}
\PYG{g+go}{   10     A    12  1010}
\PYG{g+go}{   11     B    13  1011}
\end{sphinxVerbatim}


\section{Template strings}
\label{\detokenize{string:template-strings}}\label{\detokenize{string:id1}}
Template strings provide simpler string substitutions as described in
\index{Python Enhancement Proposals@\spxentry{Python Enhancement Proposals}!PEP 292@\spxentry{PEP 292}}\sphinxhref{https://www.python.org/dev/peps/pep-0292}{\sphinxstylestrong{PEP 292}}.  A primary use case for template strings is for
internationalization (i18n) since in that context, the simpler syntax and
functionality makes it easier to translate than other built\sphinxhyphen{}in string
formatting facilities in Python.  As an example of a library built on template
strings for i18n, see the
\sphinxhref{http://flufli18n.readthedocs.io/en/latest/}{flufl.i18n} package.

\index{\$ (dollar)@\spxentry{\$}\spxextra{dollar}!in template strings@\spxentry{in template strings}}\ignorespaces
Template strings support \sphinxcode{\sphinxupquote{\$}}\sphinxhyphen{}based substitutions, using the following rules:
\begin{itemize}
\item {}
\sphinxcode{\sphinxupquote{\$\$}} is an escape; it is replaced with a single \sphinxcode{\sphinxupquote{\$}}.

\item {}
\sphinxcode{\sphinxupquote{\$identifier}} names a substitution placeholder matching a mapping key of
\sphinxcode{\sphinxupquote{"identifier"}}.  By default, \sphinxcode{\sphinxupquote{"identifier"}} is restricted to any
case\sphinxhyphen{}insensitive ASCII alphanumeric string (including underscores) that
starts with an underscore or ASCII letter.  The first non\sphinxhyphen{}identifier
character after the \sphinxcode{\sphinxupquote{\$}} character terminates this placeholder
specification.

\item {}
\sphinxcode{\sphinxupquote{\$\{identifier\}}} is equivalent to \sphinxcode{\sphinxupquote{\$identifier}}.  It is required when
valid identifier characters follow the placeholder but are not part of the
placeholder, such as \sphinxcode{\sphinxupquote{"\$\{noun\}ification"}}.

\end{itemize}

Any other appearance of \sphinxcode{\sphinxupquote{\$}} in the string will result in a \sphinxcode{\sphinxupquote{ValueError}}
being raised.

The {\hyperref[\detokenize{string:module-string}]{\sphinxcrossref{\sphinxcode{\sphinxupquote{string}}}}} module provides a {\hyperref[\detokenize{string:string.Template}]{\sphinxcrossref{\sphinxcode{\sphinxupquote{Template}}}}} class that implements
these rules.  The methods of {\hyperref[\detokenize{string:string.Template}]{\sphinxcrossref{\sphinxcode{\sphinxupquote{Template}}}}} are:
\index{Template (class in string)@\spxentry{Template}\spxextra{class in string}}

\vspace{5px}

\begin{fulllineitems}
\phantomsection\stepcounter{subsection}
\addcontentsline{toc}{subsection}{\protect\numberline{\thesubsection}{Template}}
\phantomsection\label{\detokenize{string:string.Template}}\pysiglinewithargsret{\sphinxbfcode{\sphinxupquote{class }}\sphinxcode{\sphinxupquote{string.}}\sphinxbfcode{\sphinxupquote{Template}}}{\emph{\DUrole{n}{template}}}{}
The constructor takes a single argument which is the template string.
\index{substitute() (string.Template method)@\spxentry{substitute()}\spxextra{string.Template method}}

\vspace{5px}

\begin{fulllineitems}
\phantomsection\stepcounter{subsubsection}
\addcontentsline{toc}{subsubsection}{\protect\numberline{\thesubsubsection}{substitute}}
\phantomsection\label{\detokenize{string:string.Template.substitute}}\pysiglinewithargsret{\sphinxbfcode{\sphinxupquote{substitute}}}{\emph{\DUrole{n}{mapping}\DUrole{o}{=}\DUrole{default_value}{\{\}}}, \emph{\DUrole{o}{/}}, \emph{\DUrole{o}{**}\DUrole{n}{kwds}}}{}
Performs the template substitution, returning a new string.  \sphinxstyleemphasis{mapping} is
any dictionary\sphinxhyphen{}like object with keys that match the placeholders in the
template.  Alternatively, you can provide keyword arguments, where the
keywords are the placeholders.  When both \sphinxstyleemphasis{mapping} and \sphinxstyleemphasis{kwds} are given
and there are duplicates, the placeholders from \sphinxstyleemphasis{kwds} take precedence.

\end{fulllineitems}

\index{safe\_substitute() (string.Template method)@\spxentry{safe\_substitute()}\spxextra{string.Template method}}

\vspace{5px}

\begin{fulllineitems}
\phantomsection\stepcounter{subsubsection}
\addcontentsline{toc}{subsubsection}{\protect\numberline{\thesubsubsection}{safe\_substitute}}
\phantomsection\label{\detokenize{string:string.Template.safe_substitute}}\pysiglinewithargsret{\sphinxbfcode{\sphinxupquote{safe\_substitute}}}{\emph{\DUrole{n}{mapping}\DUrole{o}{=}\DUrole{default_value}{\{\}}}, \emph{\DUrole{o}{/}}, \emph{\DUrole{o}{**}\DUrole{n}{kwds}}}{}
Like {\hyperref[\detokenize{string:string.Template.substitute}]{\sphinxcrossref{\sphinxcode{\sphinxupquote{substitute()}}}}}, except that if placeholders are missing from
\sphinxstyleemphasis{mapping} and \sphinxstyleemphasis{kwds}, instead of raising a \sphinxcode{\sphinxupquote{KeyError}} exception, the
original placeholder will appear in the resulting string intact.  Also,
unlike with {\hyperref[\detokenize{string:string.Template.substitute}]{\sphinxcrossref{\sphinxcode{\sphinxupquote{substitute()}}}}}, any other appearances of the \sphinxcode{\sphinxupquote{\$}} will
simply return \sphinxcode{\sphinxupquote{\$}} instead of raising \sphinxcode{\sphinxupquote{ValueError}}.

While other exceptions may still occur, this method is called “safe”
because it always tries to return a usable string instead of
raising an exception.  In another sense, {\hyperref[\detokenize{string:string.Template.safe_substitute}]{\sphinxcrossref{\sphinxcode{\sphinxupquote{safe\_substitute()}}}}} may be
anything other than safe, since it will silently ignore malformed
templates containing dangling delimiters, unmatched braces, or
placeholders that are not valid Python identifiers.

\end{fulllineitems}


{\hyperref[\detokenize{string:string.Template}]{\sphinxcrossref{\sphinxcode{\sphinxupquote{Template}}}}} instances also provide one public data attribute:
\index{template (string.Template attribute)@\spxentry{template}\spxextra{string.Template attribute}}

\vspace{5px}

\begin{fulllineitems}
\phantomsection\label{\detokenize{string:string.Template.template}}\pysigline{\sphinxbfcode{\sphinxupquote{template}}}
This is the object passed to the constructor’s \sphinxstyleemphasis{template} argument.  In
general, you shouldn’t change it, but read\sphinxhyphen{}only access is not enforced.

\end{fulllineitems}


\end{fulllineitems}


Here is an example of how to use a Template:

\begin{sphinxVerbatim}[commandchars=\\\{\}]
\PYG{g+gp}{\PYGZgt{}\PYGZgt{}\PYGZgt{} }\PYG{k+kn}{from} \PYG{n+nn}{string} \PYG{k+kn}{import} \PYG{n}{Template}
\PYG{g+gp}{\PYGZgt{}\PYGZgt{}\PYGZgt{} }\PYG{n}{s} \PYG{o}{=} \PYG{n}{Template}\PYG{p}{(}\PYG{l+s+s1}{\PYGZsq{}}\PYG{l+s+s1}{\PYGZdl{}who likes \PYGZdl{}what}\PYG{l+s+s1}{\PYGZsq{}}\PYG{p}{)}
\PYG{g+gp}{\PYGZgt{}\PYGZgt{}\PYGZgt{} }\PYG{n}{s}\PYG{o}{.}\PYG{n}{substitute}\PYG{p}{(}\PYG{n}{who}\PYG{o}{=}\PYG{l+s+s1}{\PYGZsq{}}\PYG{l+s+s1}{tim}\PYG{l+s+s1}{\PYGZsq{}}\PYG{p}{,} \PYG{n}{what}\PYG{o}{=}\PYG{l+s+s1}{\PYGZsq{}}\PYG{l+s+s1}{kung pao}\PYG{l+s+s1}{\PYGZsq{}}\PYG{p}{)}
\PYG{g+go}{\PYGZsq{}tim likes kung pao\PYGZsq{}}
\PYG{g+gp}{\PYGZgt{}\PYGZgt{}\PYGZgt{} }\PYG{n}{d} \PYG{o}{=} \PYG{n+nb}{dict}\PYG{p}{(}\PYG{n}{who}\PYG{o}{=}\PYG{l+s+s1}{\PYGZsq{}}\PYG{l+s+s1}{tim}\PYG{l+s+s1}{\PYGZsq{}}\PYG{p}{)}
\PYG{g+gp}{\PYGZgt{}\PYGZgt{}\PYGZgt{} }\PYG{n}{Template}\PYG{p}{(}\PYG{l+s+s1}{\PYGZsq{}}\PYG{l+s+s1}{Give \PYGZdl{}who \PYGZdl{}100}\PYG{l+s+s1}{\PYGZsq{}}\PYG{p}{)}\PYG{o}{.}\PYG{n}{substitute}\PYG{p}{(}\PYG{n}{d}\PYG{p}{)}
\PYG{g+gt}{Traceback (most recent call last):}
\PYG{c}{...}
\PYG{g+gr}{ValueError}: \PYG{n}{Invalid placeholder in string: line 1, col 11}
\PYG{g+gp}{\PYGZgt{}\PYGZgt{}\PYGZgt{} }\PYG{n}{Template}\PYG{p}{(}\PYG{l+s+s1}{\PYGZsq{}}\PYG{l+s+s1}{\PYGZdl{}who likes \PYGZdl{}what}\PYG{l+s+s1}{\PYGZsq{}}\PYG{p}{)}\PYG{o}{.}\PYG{n}{substitute}\PYG{p}{(}\PYG{n}{d}\PYG{p}{)}
\PYG{g+gt}{Traceback (most recent call last):}
\PYG{c}{...}
\PYG{g+gr}{KeyError}: \PYG{n}{\PYGZsq{}what\PYGZsq{}}
\PYG{g+gp}{\PYGZgt{}\PYGZgt{}\PYGZgt{} }\PYG{n}{Template}\PYG{p}{(}\PYG{l+s+s1}{\PYGZsq{}}\PYG{l+s+s1}{\PYGZdl{}who likes \PYGZdl{}what}\PYG{l+s+s1}{\PYGZsq{}}\PYG{p}{)}\PYG{o}{.}\PYG{n}{safe\PYGZus{}substitute}\PYG{p}{(}\PYG{n}{d}\PYG{p}{)}
\PYG{g+go}{\PYGZsq{}tim likes \PYGZdl{}what\PYGZsq{}}
\end{sphinxVerbatim}

Advanced usage: you can derive subclasses of {\hyperref[\detokenize{string:string.Template}]{\sphinxcrossref{\sphinxcode{\sphinxupquote{Template}}}}} to customize
the placeholder syntax, delimiter character, or the entire regular expression
used to parse template strings.  To do this, you can override these class
attributes:
\begin{itemize}
\item {}
\sphinxstyleemphasis{delimiter} \textendash{} This is the literal string describing a placeholder
introducing delimiter.  The default value is \sphinxcode{\sphinxupquote{\$}}.  Note that this should
\sphinxstyleemphasis{not} be a regular expression, as the implementation will call
\sphinxcode{\sphinxupquote{re.escape()}} on this string as needed.  Note further that you cannot
change the delimiter after class creation (i.e. a different delimiter must
be set in the subclass’s class namespace).

\item {}
\sphinxstyleemphasis{idpattern} \textendash{} This is the regular expression describing the pattern for
non\sphinxhyphen{}braced placeholders.  The default value is the regular expression
\sphinxcode{\sphinxupquote{(?a:{[}\_a\sphinxhyphen{}z{]}{[}\_a\sphinxhyphen{}z0\sphinxhyphen{}9{]}*)}}.  If this is given and \sphinxstyleemphasis{braceidpattern} is
\sphinxcode{\sphinxupquote{None}} this pattern will also apply to braced placeholders.

\begin{sphinxadmonition}{note}{Note:}
Since default \sphinxstyleemphasis{flags} is \sphinxcode{\sphinxupquote{re.IGNORECASE}}, pattern \sphinxcode{\sphinxupquote{{[}a\sphinxhyphen{}z{]}}} can match
with some non\sphinxhyphen{}ASCII characters. That’s why we use the local \sphinxcode{\sphinxupquote{a}} flag
here.
\end{sphinxadmonition}

\DUrole{versionmodified,changed}{Changed in version 3.7: }\sphinxstyleemphasis{braceidpattern} can be used to define separate patterns used inside and
outside the braces.

\item {}
\sphinxstyleemphasis{braceidpattern} \textendash{} This is like \sphinxstyleemphasis{idpattern} but describes the pattern for
braced placeholders.  Defaults to \sphinxcode{\sphinxupquote{None}} which means to fall back to
\sphinxstyleemphasis{idpattern} (i.e. the same pattern is used both inside and outside braces).
If given, this allows you to define different patterns for braced and
unbraced placeholders.

\DUrole{versionmodified,added}{New in version 3.7.}

\item {}
\sphinxstyleemphasis{flags} \textendash{} The regular expression flags that will be applied when compiling
the regular expression used for recognizing substitutions.  The default value
is \sphinxcode{\sphinxupquote{re.IGNORECASE}}.  Note that \sphinxcode{\sphinxupquote{re.VERBOSE}} will always be added to the
flags, so custom \sphinxstyleemphasis{idpattern}s must follow conventions for verbose regular
expressions.

\DUrole{versionmodified,added}{New in version 3.2.}

\end{itemize}

Alternatively, you can provide the entire regular expression pattern by
overriding the class attribute \sphinxstyleemphasis{pattern}.  If you do this, the value must be a
regular expression object with four named capturing groups.  The capturing
groups correspond to the rules given above, along with the invalid placeholder
rule:
\begin{itemize}
\item {}
\sphinxstyleemphasis{escaped} \textendash{} This group matches the escape sequence, e.g. \sphinxcode{\sphinxupquote{\$\$}}, in the
default pattern.

\item {}
\sphinxstyleemphasis{named} \textendash{} This group matches the unbraced placeholder name; it should not
include the delimiter in capturing group.

\item {}
\sphinxstyleemphasis{braced} \textendash{} This group matches the brace enclosed placeholder name; it should
not include either the delimiter or braces in the capturing group.

\item {}
\sphinxstyleemphasis{invalid} \textendash{} This group matches any other delimiter pattern (usually a single
delimiter), and it should appear last in the regular expression.

\end{itemize}


\section{Helper functions}
\label{\detokenize{string:helper-functions}}\index{capwords() (in module string)@\spxentry{capwords()}\spxextra{in module string}}

\vspace{5px}

\begin{fulllineitems}
\phantomsection\stepcounter{subsection}
\addcontentsline{toc}{subsection}{\protect\numberline{\thesubsection}{capwords}}
\phantomsection\label{\detokenize{string:string.capwords}}\pysiglinewithargsret{\sphinxcode{\sphinxupquote{string.}}\sphinxbfcode{\sphinxupquote{capwords}}}{\emph{\DUrole{n}{s}}, \emph{\DUrole{n}{sep}\DUrole{o}{=}\DUrole{default_value}{None}}}{}
Split the argument into words using \sphinxcode{\sphinxupquote{str.split()}}, capitalize each word
using \sphinxcode{\sphinxupquote{str.capitalize()}}, and join the capitalized words using
\sphinxcode{\sphinxupquote{str.join()}}.  If the optional second argument \sphinxstyleemphasis{sep} is absent
or \sphinxcode{\sphinxupquote{None}}, runs of whitespace characters are replaced by a single space
and leading and trailing whitespace are removed, otherwise \sphinxstyleemphasis{sep} is used to
split and join the words.

\end{fulllineitems}

\index{name\_with\_underscores (built\sphinxhyphen{}in class)@\spxentry{name\_with\_underscores}\spxextra{built\sphinxhyphen{}in class}}

\vspace{5px}

\begin{fulllineitems}
\phantomsection\stepcounter{chapter}
\addcontentsline{toc}{chapter}{\protect\numberline{\thechapter}{name\_with\_underscores}}
\phantomsection\label{\detokenize{index:name_with_underscores}}\pysigline{\sphinxbfcode{\sphinxupquote{class }}\sphinxbfcode{\sphinxupquote{name\_with\_underscores}}}
\end{fulllineitems}



\renewcommand{\indexname}{Python Module Index}
\begin{sphinxtheindex}
\let\bigletter\sphinxstyleindexlettergroup
\bigletter{c}
\item\relax\sphinxstyleindexentry{csv}\sphinxstyleindexpageref{csv:\detokenize{module-csv}}
\indexspace
\bigletter{s}
\item\relax\sphinxstyleindexentry{string}\sphinxstyleindexpageref{string:\detokenize{module-string}}
\end{sphinxtheindex}

\renewcommand{\indexname}{Index}
\printindex
\end{document}
